\documentclass[11pt]{article}

    \usepackage[breakable]{tcolorbox}
    \usepackage{parskip} % Stop auto-indenting (to mimic markdown behaviour)
    

    % Basic figure setup, for now with no caption control since it's done
    % automatically by Pandoc (which extracts ![](path) syntax from Markdown).
    \usepackage{graphicx}
    % Maintain compatibility with old templates. Remove in nbconvert 6.0
    \let\Oldincludegraphics\includegraphics
    % Ensure that by default, figures have no caption (until we provide a
    % proper Figure object with a Caption API and a way to capture that
    % in the conversion process - todo).
    \usepackage{caption}
    \DeclareCaptionFormat{nocaption}{}
    \captionsetup{format=nocaption,aboveskip=0pt,belowskip=0pt}

    \usepackage{float}
    \floatplacement{figure}{H} % forces figures to be placed at the correct location
    \usepackage{xcolor} % Allow colors to be defined
    \usepackage{enumerate} % Needed for markdown enumerations to work
    \usepackage{geometry} % Used to adjust the document margins
    \usepackage{amsmath} % Equations
    \usepackage{amssymb} % Equations
    \usepackage{textcomp} % defines textquotesingle
    % Hack from http://tex.stackexchange.com/a/47451/13684:
    \AtBeginDocument{%
        \def\PYZsq{\textquotesingle}% Upright quotes in Pygmentized code
    }
    \usepackage{upquote} % Upright quotes for verbatim code
    \usepackage{eurosym} % defines \euro

    \usepackage{iftex}
    \ifPDFTeX
        \usepackage[T1]{fontenc}
        \IfFileExists{alphabeta.sty}{
              \usepackage{alphabeta}
          }{
              \usepackage[mathletters]{ucs}
              \usepackage[utf8x]{inputenc}
          }
    \else
        \usepackage{fontspec}
        \usepackage{unicode-math}
    \fi

    \usepackage{fancyvrb} % verbatim replacement that allows latex
    \usepackage{grffile} % extends the file name processing of package graphics
                         % to support a larger range
    \makeatletter % fix for old versions of grffile with XeLaTeX
    \@ifpackagelater{grffile}{2019/11/01}
    {
      % Do nothing on new versions
    }
    {
      \def\Gread@@xetex#1{%
        \IfFileExists{"\Gin@base".bb}%
        {\Gread@eps{\Gin@base.bb}}%
        {\Gread@@xetex@aux#1}%
      }
    }
    \makeatother
    \usepackage[Export]{adjustbox} % Used to constrain images to a maximum size
    \adjustboxset{max size={0.9\linewidth}{0.9\paperheight}}

    % The hyperref package gives us a pdf with properly built
    % internal navigation ('pdf bookmarks' for the table of contents,
    % internal cross-reference links, web links for URLs, etc.)
    \usepackage{hyperref}
    % The default LaTeX title has an obnoxious amount of whitespace. By default,
    % titling removes some of it. It also provides customization options.
    \usepackage{titling}
    \usepackage{longtable} % longtable support required by pandoc >1.10
    \usepackage{booktabs}  % table support for pandoc > 1.12.2
    \usepackage{array}     % table support for pandoc >= 2.11.3
    \usepackage{calc}      % table minipage width calculation for pandoc >= 2.11.1
    \usepackage[inline]{enumitem} % IRkernel/repr support (it uses the enumerate* environment)
    \usepackage[normalem]{ulem} % ulem is needed to support strikethroughs (\sout)
                                % normalem makes italics be italics, not underlines
    \usepackage{mathrsfs}
    

    
    % Colors for the hyperref package
    \definecolor{urlcolor}{rgb}{0,.145,.698}
    \definecolor{linkcolor}{rgb}{.71,0.21,0.01}
    \definecolor{citecolor}{rgb}{.12,.54,.11}

    % ANSI colors
    \definecolor{ansi-black}{HTML}{3E424D}
    \definecolor{ansi-black-intense}{HTML}{282C36}
    \definecolor{ansi-red}{HTML}{E75C58}
    \definecolor{ansi-red-intense}{HTML}{B22B31}
    \definecolor{ansi-green}{HTML}{00A250}
    \definecolor{ansi-green-intense}{HTML}{007427}
    \definecolor{ansi-yellow}{HTML}{DDB62B}
    \definecolor{ansi-yellow-intense}{HTML}{B27D12}
    \definecolor{ansi-blue}{HTML}{208FFB}
    \definecolor{ansi-blue-intense}{HTML}{0065CA}
    \definecolor{ansi-magenta}{HTML}{D160C4}
    \definecolor{ansi-magenta-intense}{HTML}{A03196}
    \definecolor{ansi-cyan}{HTML}{60C6C8}
    \definecolor{ansi-cyan-intense}{HTML}{258F8F}
    \definecolor{ansi-white}{HTML}{C5C1B4}
    \definecolor{ansi-white-intense}{HTML}{A1A6B2}
    \definecolor{ansi-default-inverse-fg}{HTML}{FFFFFF}
    \definecolor{ansi-default-inverse-bg}{HTML}{000000}

    % common color for the border for error outputs.
    \definecolor{outerrorbackground}{HTML}{FFDFDF}

    % commands and environments needed by pandoc snippets
    % extracted from the output of `pandoc -s`
    \providecommand{\tightlist}{%
      \setlength{\itemsep}{0pt}\setlength{\parskip}{0pt}}
    \DefineVerbatimEnvironment{Highlighting}{Verbatim}{commandchars=\\\{\}}
    % Add ',fontsize=\small' for more characters per line
    \newenvironment{Shaded}{}{}
    \newcommand{\KeywordTok}[1]{\textcolor[rgb]{0.00,0.44,0.13}{\textbf{{#1}}}}
    \newcommand{\DataTypeTok}[1]{\textcolor[rgb]{0.56,0.13,0.00}{{#1}}}
    \newcommand{\DecValTok}[1]{\textcolor[rgb]{0.25,0.63,0.44}{{#1}}}
    \newcommand{\BaseNTok}[1]{\textcolor[rgb]{0.25,0.63,0.44}{{#1}}}
    \newcommand{\FloatTok}[1]{\textcolor[rgb]{0.25,0.63,0.44}{{#1}}}
    \newcommand{\CharTok}[1]{\textcolor[rgb]{0.25,0.44,0.63}{{#1}}}
    \newcommand{\StringTok}[1]{\textcolor[rgb]{0.25,0.44,0.63}{{#1}}}
    \newcommand{\CommentTok}[1]{\textcolor[rgb]{0.38,0.63,0.69}{\textit{{#1}}}}
    \newcommand{\OtherTok}[1]{\textcolor[rgb]{0.00,0.44,0.13}{{#1}}}
    \newcommand{\AlertTok}[1]{\textcolor[rgb]{1.00,0.00,0.00}{\textbf{{#1}}}}
    \newcommand{\FunctionTok}[1]{\textcolor[rgb]{0.02,0.16,0.49}{{#1}}}
    \newcommand{\RegionMarkerTok}[1]{{#1}}
    \newcommand{\ErrorTok}[1]{\textcolor[rgb]{1.00,0.00,0.00}{\textbf{{#1}}}}
    \newcommand{\NormalTok}[1]{{#1}}

    % Additional commands for more recent versions of Pandoc
    \newcommand{\ConstantTok}[1]{\textcolor[rgb]{0.53,0.00,0.00}{{#1}}}
    \newcommand{\SpecialCharTok}[1]{\textcolor[rgb]{0.25,0.44,0.63}{{#1}}}
    \newcommand{\VerbatimStringTok}[1]{\textcolor[rgb]{0.25,0.44,0.63}{{#1}}}
    \newcommand{\SpecialStringTok}[1]{\textcolor[rgb]{0.73,0.40,0.53}{{#1}}}
    \newcommand{\ImportTok}[1]{{#1}}
    \newcommand{\DocumentationTok}[1]{\textcolor[rgb]{0.73,0.13,0.13}{\textit{{#1}}}}
    \newcommand{\AnnotationTok}[1]{\textcolor[rgb]{0.38,0.63,0.69}{\textbf{\textit{{#1}}}}}
    \newcommand{\CommentVarTok}[1]{\textcolor[rgb]{0.38,0.63,0.69}{\textbf{\textit{{#1}}}}}
    \newcommand{\VariableTok}[1]{\textcolor[rgb]{0.10,0.09,0.49}{{#1}}}
    \newcommand{\ControlFlowTok}[1]{\textcolor[rgb]{0.00,0.44,0.13}{\textbf{{#1}}}}
    \newcommand{\OperatorTok}[1]{\textcolor[rgb]{0.40,0.40,0.40}{{#1}}}
    \newcommand{\BuiltInTok}[1]{{#1}}
    \newcommand{\ExtensionTok}[1]{{#1}}
    \newcommand{\PreprocessorTok}[1]{\textcolor[rgb]{0.74,0.48,0.00}{{#1}}}
    \newcommand{\AttributeTok}[1]{\textcolor[rgb]{0.49,0.56,0.16}{{#1}}}
    \newcommand{\InformationTok}[1]{\textcolor[rgb]{0.38,0.63,0.69}{\textbf{\textit{{#1}}}}}
    \newcommand{\WarningTok}[1]{\textcolor[rgb]{0.38,0.63,0.69}{\textbf{\textit{{#1}}}}}


    % Define a nice break command that doesn't care if a line doesn't already
    % exist.
    \def\br{\hspace*{\fill} \\* }
    % Math Jax compatibility definitions
    \def\gt{>}
    \def\lt{<}
    \let\Oldtex\TeX
    \let\Oldlatex\LaTeX
    \renewcommand{\TeX}{\textrm{\Oldtex}}
    \renewcommand{\LaTeX}{\textrm{\Oldlatex}}
    % Document parameters
    % Document title
    \title{Diabetes risk factors}
    
    
    
    
    
    \author{Viktória Nemkin (M8GXSS)}
    
    
    
% Pygments definitions
\makeatletter
\def\PY@reset{\let\PY@it=\relax \let\PY@bf=\relax%
    \let\PY@ul=\relax \let\PY@tc=\relax%
    \let\PY@bc=\relax \let\PY@ff=\relax}
\def\PY@tok#1{\csname PY@tok@#1\endcsname}
\def\PY@toks#1+{\ifx\relax#1\empty\else%
    \PY@tok{#1}\expandafter\PY@toks\fi}
\def\PY@do#1{\PY@bc{\PY@tc{\PY@ul{%
    \PY@it{\PY@bf{\PY@ff{#1}}}}}}}
\def\PY#1#2{\PY@reset\PY@toks#1+\relax+\PY@do{#2}}

\@namedef{PY@tok@w}{\def\PY@tc##1{\textcolor[rgb]{0.73,0.73,0.73}{##1}}}
\@namedef{PY@tok@c}{\let\PY@it=\textit\def\PY@tc##1{\textcolor[rgb]{0.24,0.48,0.48}{##1}}}
\@namedef{PY@tok@cp}{\def\PY@tc##1{\textcolor[rgb]{0.61,0.40,0.00}{##1}}}
\@namedef{PY@tok@k}{\let\PY@bf=\textbf\def\PY@tc##1{\textcolor[rgb]{0.00,0.50,0.00}{##1}}}
\@namedef{PY@tok@kp}{\def\PY@tc##1{\textcolor[rgb]{0.00,0.50,0.00}{##1}}}
\@namedef{PY@tok@kt}{\def\PY@tc##1{\textcolor[rgb]{0.69,0.00,0.25}{##1}}}
\@namedef{PY@tok@o}{\def\PY@tc##1{\textcolor[rgb]{0.40,0.40,0.40}{##1}}}
\@namedef{PY@tok@ow}{\let\PY@bf=\textbf\def\PY@tc##1{\textcolor[rgb]{0.67,0.13,1.00}{##1}}}
\@namedef{PY@tok@nb}{\def\PY@tc##1{\textcolor[rgb]{0.00,0.50,0.00}{##1}}}
\@namedef{PY@tok@nf}{\def\PY@tc##1{\textcolor[rgb]{0.00,0.00,1.00}{##1}}}
\@namedef{PY@tok@nc}{\let\PY@bf=\textbf\def\PY@tc##1{\textcolor[rgb]{0.00,0.00,1.00}{##1}}}
\@namedef{PY@tok@nn}{\let\PY@bf=\textbf\def\PY@tc##1{\textcolor[rgb]{0.00,0.00,1.00}{##1}}}
\@namedef{PY@tok@ne}{\let\PY@bf=\textbf\def\PY@tc##1{\textcolor[rgb]{0.80,0.25,0.22}{##1}}}
\@namedef{PY@tok@nv}{\def\PY@tc##1{\textcolor[rgb]{0.10,0.09,0.49}{##1}}}
\@namedef{PY@tok@no}{\def\PY@tc##1{\textcolor[rgb]{0.53,0.00,0.00}{##1}}}
\@namedef{PY@tok@nl}{\def\PY@tc##1{\textcolor[rgb]{0.46,0.46,0.00}{##1}}}
\@namedef{PY@tok@ni}{\let\PY@bf=\textbf\def\PY@tc##1{\textcolor[rgb]{0.44,0.44,0.44}{##1}}}
\@namedef{PY@tok@na}{\def\PY@tc##1{\textcolor[rgb]{0.41,0.47,0.13}{##1}}}
\@namedef{PY@tok@nt}{\let\PY@bf=\textbf\def\PY@tc##1{\textcolor[rgb]{0.00,0.50,0.00}{##1}}}
\@namedef{PY@tok@nd}{\def\PY@tc##1{\textcolor[rgb]{0.67,0.13,1.00}{##1}}}
\@namedef{PY@tok@s}{\def\PY@tc##1{\textcolor[rgb]{0.73,0.13,0.13}{##1}}}
\@namedef{PY@tok@sd}{\let\PY@it=\textit\def\PY@tc##1{\textcolor[rgb]{0.73,0.13,0.13}{##1}}}
\@namedef{PY@tok@si}{\let\PY@bf=\textbf\def\PY@tc##1{\textcolor[rgb]{0.64,0.35,0.47}{##1}}}
\@namedef{PY@tok@se}{\let\PY@bf=\textbf\def\PY@tc##1{\textcolor[rgb]{0.67,0.36,0.12}{##1}}}
\@namedef{PY@tok@sr}{\def\PY@tc##1{\textcolor[rgb]{0.64,0.35,0.47}{##1}}}
\@namedef{PY@tok@ss}{\def\PY@tc##1{\textcolor[rgb]{0.10,0.09,0.49}{##1}}}
\@namedef{PY@tok@sx}{\def\PY@tc##1{\textcolor[rgb]{0.00,0.50,0.00}{##1}}}
\@namedef{PY@tok@m}{\def\PY@tc##1{\textcolor[rgb]{0.40,0.40,0.40}{##1}}}
\@namedef{PY@tok@gh}{\let\PY@bf=\textbf\def\PY@tc##1{\textcolor[rgb]{0.00,0.00,0.50}{##1}}}
\@namedef{PY@tok@gu}{\let\PY@bf=\textbf\def\PY@tc##1{\textcolor[rgb]{0.50,0.00,0.50}{##1}}}
\@namedef{PY@tok@gd}{\def\PY@tc##1{\textcolor[rgb]{0.63,0.00,0.00}{##1}}}
\@namedef{PY@tok@gi}{\def\PY@tc##1{\textcolor[rgb]{0.00,0.52,0.00}{##1}}}
\@namedef{PY@tok@gr}{\def\PY@tc##1{\textcolor[rgb]{0.89,0.00,0.00}{##1}}}
\@namedef{PY@tok@ge}{\let\PY@it=\textit}
\@namedef{PY@tok@gs}{\let\PY@bf=\textbf}
\@namedef{PY@tok@gp}{\let\PY@bf=\textbf\def\PY@tc##1{\textcolor[rgb]{0.00,0.00,0.50}{##1}}}
\@namedef{PY@tok@go}{\def\PY@tc##1{\textcolor[rgb]{0.44,0.44,0.44}{##1}}}
\@namedef{PY@tok@gt}{\def\PY@tc##1{\textcolor[rgb]{0.00,0.27,0.87}{##1}}}
\@namedef{PY@tok@err}{\def\PY@bc##1{{\setlength{\fboxsep}{\string -\fboxrule}\fcolorbox[rgb]{1.00,0.00,0.00}{1,1,1}{\strut ##1}}}}
\@namedef{PY@tok@kc}{\let\PY@bf=\textbf\def\PY@tc##1{\textcolor[rgb]{0.00,0.50,0.00}{##1}}}
\@namedef{PY@tok@kd}{\let\PY@bf=\textbf\def\PY@tc##1{\textcolor[rgb]{0.00,0.50,0.00}{##1}}}
\@namedef{PY@tok@kn}{\let\PY@bf=\textbf\def\PY@tc##1{\textcolor[rgb]{0.00,0.50,0.00}{##1}}}
\@namedef{PY@tok@kr}{\let\PY@bf=\textbf\def\PY@tc##1{\textcolor[rgb]{0.00,0.50,0.00}{##1}}}
\@namedef{PY@tok@bp}{\def\PY@tc##1{\textcolor[rgb]{0.00,0.50,0.00}{##1}}}
\@namedef{PY@tok@fm}{\def\PY@tc##1{\textcolor[rgb]{0.00,0.00,1.00}{##1}}}
\@namedef{PY@tok@vc}{\def\PY@tc##1{\textcolor[rgb]{0.10,0.09,0.49}{##1}}}
\@namedef{PY@tok@vg}{\def\PY@tc##1{\textcolor[rgb]{0.10,0.09,0.49}{##1}}}
\@namedef{PY@tok@vi}{\def\PY@tc##1{\textcolor[rgb]{0.10,0.09,0.49}{##1}}}
\@namedef{PY@tok@vm}{\def\PY@tc##1{\textcolor[rgb]{0.10,0.09,0.49}{##1}}}
\@namedef{PY@tok@sa}{\def\PY@tc##1{\textcolor[rgb]{0.73,0.13,0.13}{##1}}}
\@namedef{PY@tok@sb}{\def\PY@tc##1{\textcolor[rgb]{0.73,0.13,0.13}{##1}}}
\@namedef{PY@tok@sc}{\def\PY@tc##1{\textcolor[rgb]{0.73,0.13,0.13}{##1}}}
\@namedef{PY@tok@dl}{\def\PY@tc##1{\textcolor[rgb]{0.73,0.13,0.13}{##1}}}
\@namedef{PY@tok@s2}{\def\PY@tc##1{\textcolor[rgb]{0.73,0.13,0.13}{##1}}}
\@namedef{PY@tok@sh}{\def\PY@tc##1{\textcolor[rgb]{0.73,0.13,0.13}{##1}}}
\@namedef{PY@tok@s1}{\def\PY@tc##1{\textcolor[rgb]{0.73,0.13,0.13}{##1}}}
\@namedef{PY@tok@mb}{\def\PY@tc##1{\textcolor[rgb]{0.40,0.40,0.40}{##1}}}
\@namedef{PY@tok@mf}{\def\PY@tc##1{\textcolor[rgb]{0.40,0.40,0.40}{##1}}}
\@namedef{PY@tok@mh}{\def\PY@tc##1{\textcolor[rgb]{0.40,0.40,0.40}{##1}}}
\@namedef{PY@tok@mi}{\def\PY@tc##1{\textcolor[rgb]{0.40,0.40,0.40}{##1}}}
\@namedef{PY@tok@il}{\def\PY@tc##1{\textcolor[rgb]{0.40,0.40,0.40}{##1}}}
\@namedef{PY@tok@mo}{\def\PY@tc##1{\textcolor[rgb]{0.40,0.40,0.40}{##1}}}
\@namedef{PY@tok@ch}{\let\PY@it=\textit\def\PY@tc##1{\textcolor[rgb]{0.24,0.48,0.48}{##1}}}
\@namedef{PY@tok@cm}{\let\PY@it=\textit\def\PY@tc##1{\textcolor[rgb]{0.24,0.48,0.48}{##1}}}
\@namedef{PY@tok@cpf}{\let\PY@it=\textit\def\PY@tc##1{\textcolor[rgb]{0.24,0.48,0.48}{##1}}}
\@namedef{PY@tok@c1}{\let\PY@it=\textit\def\PY@tc##1{\textcolor[rgb]{0.24,0.48,0.48}{##1}}}
\@namedef{PY@tok@cs}{\let\PY@it=\textit\def\PY@tc##1{\textcolor[rgb]{0.24,0.48,0.48}{##1}}}

\def\PYZbs{\char`\\}
\def\PYZus{\char`\_}
\def\PYZob{\char`\{}
\def\PYZcb{\char`\}}
\def\PYZca{\char`\^}
\def\PYZam{\char`\&}
\def\PYZlt{\char`\<}
\def\PYZgt{\char`\>}
\def\PYZsh{\char`\#}
\def\PYZpc{\char`\%}
\def\PYZdl{\char`\$}
\def\PYZhy{\char`\-}
\def\PYZsq{\char`\'}
\def\PYZdq{\char`\"}
\def\PYZti{\char`\~}
% for compatibility with earlier versions
\def\PYZat{@}
\def\PYZlb{[}
\def\PYZrb{]}
\makeatother


    % For linebreaks inside Verbatim environment from package fancyvrb.
    \makeatletter
        \newbox\Wrappedcontinuationbox
        \newbox\Wrappedvisiblespacebox
        \newcommand*\Wrappedvisiblespace {\textcolor{red}{\textvisiblespace}}
        \newcommand*\Wrappedcontinuationsymbol {\textcolor{red}{\llap{\tiny$\m@th\hookrightarrow$}}}
        \newcommand*\Wrappedcontinuationindent {3ex }
        \newcommand*\Wrappedafterbreak {\kern\Wrappedcontinuationindent\copy\Wrappedcontinuationbox}
        % Take advantage of the already applied Pygments mark-up to insert
        % potential linebreaks for TeX processing.
        %        {, <, #, %, $, ' and ": go to next line.
        %        _, }, ^, &, >, - and ~: stay at end of broken line.
        % Use of \textquotesingle for straight quote.
        \newcommand*\Wrappedbreaksatspecials {%
            \def\PYGZus{\discretionary{\char`\_}{\Wrappedafterbreak}{\char`\_}}%
            \def\PYGZob{\discretionary{}{\Wrappedafterbreak\char`\{}{\char`\{}}%
            \def\PYGZcb{\discretionary{\char`\}}{\Wrappedafterbreak}{\char`\}}}%
            \def\PYGZca{\discretionary{\char`\^}{\Wrappedafterbreak}{\char`\^}}%
            \def\PYGZam{\discretionary{\char`\&}{\Wrappedafterbreak}{\char`\&}}%
            \def\PYGZlt{\discretionary{}{\Wrappedafterbreak\char`\<}{\char`\<}}%
            \def\PYGZgt{\discretionary{\char`\>}{\Wrappedafterbreak}{\char`\>}}%
            \def\PYGZsh{\discretionary{}{\Wrappedafterbreak\char`\#}{\char`\#}}%
            \def\PYGZpc{\discretionary{}{\Wrappedafterbreak\char`\%}{\char`\%}}%
            \def\PYGZdl{\discretionary{}{\Wrappedafterbreak\char`\$}{\char`\$}}%
            \def\PYGZhy{\discretionary{\char`\-}{\Wrappedafterbreak}{\char`\-}}%
            \def\PYGZsq{\discretionary{}{\Wrappedafterbreak\textquotesingle}{\textquotesingle}}%
            \def\PYGZdq{\discretionary{}{\Wrappedafterbreak\char`\"}{\char`\"}}%
            \def\PYGZti{\discretionary{\char`\~}{\Wrappedafterbreak}{\char`\~}}%
        }
        % Some characters . , ; ? ! / are not pygmentized.
        % This macro makes them "active" and they will insert potential linebreaks
        \newcommand*\Wrappedbreaksatpunct {%
            \lccode`\~`\.\lowercase{\def~}{\discretionary{\hbox{\char`\.}}{\Wrappedafterbreak}{\hbox{\char`\.}}}%
            \lccode`\~`\,\lowercase{\def~}{\discretionary{\hbox{\char`\,}}{\Wrappedafterbreak}{\hbox{\char`\,}}}%
            \lccode`\~`\;\lowercase{\def~}{\discretionary{\hbox{\char`\;}}{\Wrappedafterbreak}{\hbox{\char`\;}}}%
            \lccode`\~`\:\lowercase{\def~}{\discretionary{\hbox{\char`\:}}{\Wrappedafterbreak}{\hbox{\char`\:}}}%
            \lccode`\~`\?\lowercase{\def~}{\discretionary{\hbox{\char`\?}}{\Wrappedafterbreak}{\hbox{\char`\?}}}%
            \lccode`\~`\!\lowercase{\def~}{\discretionary{\hbox{\char`\!}}{\Wrappedafterbreak}{\hbox{\char`\!}}}%
            \lccode`\~`\/\lowercase{\def~}{\discretionary{\hbox{\char`\/}}{\Wrappedafterbreak}{\hbox{\char`\/}}}%
            \catcode`\.\active
            \catcode`\,\active
            \catcode`\;\active
            \catcode`\:\active
            \catcode`\?\active
            \catcode`\!\active
            \catcode`\/\active
            \lccode`\~`\~
        }
    \makeatother

    \let\OriginalVerbatim=\Verbatim
    \makeatletter
    \renewcommand{\Verbatim}[1][1]{%
        %\parskip\z@skip
        \sbox\Wrappedcontinuationbox {\Wrappedcontinuationsymbol}%
        \sbox\Wrappedvisiblespacebox {\FV@SetupFont\Wrappedvisiblespace}%
        \def\FancyVerbFormatLine ##1{\hsize\linewidth
            \vtop{\raggedright\hyphenpenalty\z@\exhyphenpenalty\z@
                \doublehyphendemerits\z@\finalhyphendemerits\z@
                \strut ##1\strut}%
        }%
        % If the linebreak is at a space, the latter will be displayed as visible
        % space at end of first line, and a continuation symbol starts next line.
        % Stretch/shrink are however usually zero for typewriter font.
        \def\FV@Space {%
            \nobreak\hskip\z@ plus\fontdimen3\font minus\fontdimen4\font
            \discretionary{\copy\Wrappedvisiblespacebox}{\Wrappedafterbreak}
            {\kern\fontdimen2\font}%
        }%

        % Allow breaks at special characters using \PYG... macros.
        \Wrappedbreaksatspecials
        % Breaks at punctuation characters . , ; ? ! and / need catcode=\active
        \OriginalVerbatim[#1,codes*=\Wrappedbreaksatpunct]%
    }
    \makeatother

    % Exact colors from NB
    \definecolor{incolor}{HTML}{303F9F}
    \definecolor{outcolor}{HTML}{D84315}
    \definecolor{cellborder}{HTML}{CFCFCF}
    \definecolor{cellbackground}{HTML}{F7F7F7}

    % prompt
    \makeatletter
    \newcommand{\boxspacing}{\kern\kvtcb@left@rule\kern\kvtcb@boxsep}
    \makeatother
    \newcommand{\prompt}[4]{
        {\ttfamily\llap{{\color{#2}[#3]:\hspace{3pt}#4}}\vspace{-\baselineskip}}
    }
    

    
    % Prevent overflowing lines due to hard-to-break entities
    \sloppy
    % Setup hyperref package
    \hypersetup{
      breaklinks=true,  % so long urls are correctly broken across lines
      colorlinks=true,
      urlcolor=urlcolor,
      linkcolor=linkcolor,
      citecolor=citecolor,
      }
    % Slightly bigger margins than the latex defaults
    
    \geometry{verbose,tmargin=1in,bmargin=1in,lmargin=1in,rmargin=1in}
    
    

\begin{document}
    
    \maketitle
    \tableofcontents
    

    
    \hypertarget{setup}{%
\section{Setup}\label{setup}}

I am interested in medical research, so I have chosen the
\href{https://www.kaggle.com/datasets/iammustafatz/diabetes-prediction-dataset}{Diabetes
prediction dataset} from Kaggle as the topic of my homework project. The
goal of this dataset is to predict whether someone will develop
diabetes, based on key indicators of the disease.

    \hypertarget{input-data}{%
\subsection{Input data}\label{input-data}}

The dataset is anonymised and contains the following data about 100,000
individuals:

\begin{itemize}
\tightlist
\item
  \textbf{Age}: Ranges from 0-80, diabetes is more common amongst older
  adults.
\item
  \textbf{Gender}: Can also have an impact on a person's susceptibility.
\item
  \textbf{Body Mass Index (BMI)}: Higher BMI values are linked to higher
  diabetes risk.
\item
  \textbf{Hypertension}: Presistently elevated blood pressure in the
  arteries, linked to heart disease.
\item
  \textbf{Heart disease}: Associated with a risk of developing diabetes.
\item
  \textbf{Smoking history}: Considered as a risk factor, can worsen the
  complications of diabetes.
\item
  \textbf{HbA1c level}: Hemoglobin A1c, measures blood sugar level over
  the past 2-3 months. Over 6.5\% indicates diabetes.
\item
  \textbf{Blood glucose level}: Key indicator of diabetes.
\item
  \textbf{Diabetes}: Target value.
\end{itemize}

These are some of the key indicators of diabetes, along with demographic
data, which could be used to determine risk factors for developing
diabetes.

While it is not explicitly stated, the data is definitely on Type 2
diabetes, since Type 1 is a genetic condition.

    \hypertarget{tools}{%
\subsection{Tools}\label{tools}}

I used Python, the \texttt{numpy} and \texttt{pandas} libraries for
manipulation of the dataset, \texttt{matplotlib}, \texttt{plotly} and
\texttt{seaborn} for plotting and visualising and \texttt{scipy} and
\texttt{scikit-learn} for the various statistical analysis and
evaluation tools they offer.

    \begin{tcolorbox}[breakable, size=fbox, boxrule=1pt, pad at break*=1mm,colback=cellbackground, colframe=cellborder]
\prompt{In}{incolor}{220}{\boxspacing}
\begin{Verbatim}[commandchars=\\\{\}]
\PY{k+kn}{import} \PY{n+nn}{numpy} \PY{k}{as} \PY{n+nn}{np}
\PY{k+kn}{import} \PY{n+nn}{pandas} \PY{k}{as} \PY{n+nn}{pd}
\PY{k+kn}{import} \PY{n+nn}{matplotlib}\PY{n+nn}{.}\PY{n+nn}{pyplot} \PY{k}{as} \PY{n+nn}{plt}
\PY{k+kn}{import} \PY{n+nn}{matplotlib}\PY{n+nn}{.}\PY{n+nn}{cm} \PY{k}{as} \PY{n+nn}{cm}
\PY{k+kn}{import} \PY{n+nn}{matplotlib}\PY{n+nn}{.}\PY{n+nn}{colors} \PY{k}{as} \PY{n+nn}{colors}
\PY{k+kn}{import} \PY{n+nn}{seaborn} \PY{k}{as} \PY{n+nn}{sns}
\PY{k+kn}{import} \PY{n+nn}{plotly}\PY{n+nn}{.}\PY{n+nn}{express} \PY{k}{as} \PY{n+nn}{px}
\PY{k+kn}{from} \PY{n+nn}{itertools} \PY{k+kn}{import} \PY{n}{chain}\PY{p}{,} \PY{n}{combinations}
\PY{k+kn}{from} \PY{n+nn}{sklearn}\PY{n+nn}{.}\PY{n+nn}{linear\PYZus{}model} \PY{k+kn}{import} \PY{n}{LinearRegression}
\PY{k+kn}{from} \PY{n+nn}{sklearn}\PY{n+nn}{.}\PY{n+nn}{model\PYZus{}selection} \PY{k+kn}{import} \PY{n}{train\PYZus{}test\PYZus{}split}
\PY{k+kn}{from} \PY{n+nn}{sklearn}\PY{n+nn}{.}\PY{n+nn}{metrics} \PY{k+kn}{import} \PY{n}{mean\PYZus{}squared\PYZus{}error}
\PY{k+kn}{from} \PY{n+nn}{scipy}\PY{n+nn}{.}\PY{n+nn}{stats} \PY{k+kn}{import} \PY{n}{chi2\PYZus{}contingency}
\PY{k+kn}{from} \PY{n+nn}{scipy}\PY{n+nn}{.}\PY{n+nn}{stats} \PY{k+kn}{import} \PY{n}{ttest\PYZus{}ind}
\end{Verbatim}
\end{tcolorbox}

    \hypertarget{source-code-availability}{%
\subsection{Source code availability}\label{source-code-availability}}

The code and the dataset is available in my Github repository here:

https://github.com/nemkin/matstat-diabetes

    \hypertarget{data-cleaning-and-sanity-checks}{%
\subsection{Data cleaning and sanity
checks}\label{data-cleaning-and-sanity-checks}}

The first step in working with data is making sure it is not flawed.
There is no single method that we can follow and guarantee no issues
persist, but essentially we need to be careful and thourough and make
sure nothing weird is happening in the dataset.

First, I read in the dataset from the csv file.

    \begin{tcolorbox}[breakable, size=fbox, boxrule=1pt, pad at break*=1mm,colback=cellbackground, colframe=cellborder]
\prompt{In}{incolor}{221}{\boxspacing}
\begin{Verbatim}[commandchars=\\\{\}]
\PY{n}{df} \PY{o}{=} \PY{n}{pd}\PY{o}{.}\PY{n}{read\PYZus{}csv}\PY{p}{(}\PY{l+s+s1}{\PYZsq{}}\PY{l+s+s1}{../data/diabetes\PYZus{}prediction\PYZus{}dataset.csv}\PY{l+s+s1}{\PYZsq{}}\PY{p}{)}
\PY{n}{df}\PY{o}{.}\PY{n}{head}\PY{p}{(}\PY{l+m+mi}{10}\PY{p}{)}
\end{Verbatim}
\end{tcolorbox}

            \begin{tcolorbox}[breakable, size=fbox, boxrule=.5pt, pad at break*=1mm, opacityfill=0]
\prompt{Out}{outcolor}{221}{\boxspacing}
\begin{Verbatim}[commandchars=\\\{\}]
   gender   age  hypertension  heart\_disease smoking\_history    bmi
0  Female  80.0             0              1           never  25.19  \textbackslash{}
1  Female  54.0             0              0         No Info  27.32
2    Male  28.0             0              0           never  27.32
3  Female  36.0             0              0         current  23.45
4    Male  76.0             1              1         current  20.14
5  Female  20.0             0              0           never  27.32
6  Female  44.0             0              0           never  19.31
7  Female  79.0             0              0         No Info  23.86
8    Male  42.0             0              0           never  33.64
9  Female  32.0             0              0           never  27.32

   HbA1c\_level  blood\_glucose\_level  diabetes
0          6.6                  140         0
1          6.6                   80         0
2          5.7                  158         0
3          5.0                  155         0
4          4.8                  155         0
5          6.6                   85         0
6          6.5                  200         1
7          5.7                   85         0
8          4.8                  145         0
9          5.0                  100         0
\end{Verbatim}
\end{tcolorbox}
        
    I like check the value ranges for the data and make sure, that the
columns are correctly typed:

    \begin{tcolorbox}[breakable, size=fbox, boxrule=1pt, pad at break*=1mm,colback=cellbackground, colframe=cellborder]
\prompt{In}{incolor}{222}{\boxspacing}
\begin{Verbatim}[commandchars=\\\{\}]
\PY{n}{df}\PY{o}{.}\PY{n}{dtypes}
\end{Verbatim}
\end{tcolorbox}

            \begin{tcolorbox}[breakable, size=fbox, boxrule=.5pt, pad at break*=1mm, opacityfill=0]
\prompt{Out}{outcolor}{222}{\boxspacing}
\begin{Verbatim}[commandchars=\\\{\}]
gender                  object
age                    float64
hypertension             int64
heart\_disease            int64
smoking\_history         object
bmi                    float64
HbA1c\_level            float64
blood\_glucose\_level      int64
diabetes                 int64
dtype: object
\end{Verbatim}
\end{tcolorbox}
        
    We can use the describe method to check the value ranges of the columns:

    \begin{tcolorbox}[breakable, size=fbox, boxrule=1pt, pad at break*=1mm,colback=cellbackground, colframe=cellborder]
\prompt{In}{incolor}{223}{\boxspacing}
\begin{Verbatim}[commandchars=\\\{\}]
\PY{n}{df}\PY{o}{.}\PY{n}{describe}\PY{p}{(}\PY{n}{include}\PY{o}{=}\PY{l+s+s1}{\PYZsq{}}\PY{l+s+s1}{all}\PY{l+s+s1}{\PYZsq{}}\PY{p}{)}
\end{Verbatim}
\end{tcolorbox}

            \begin{tcolorbox}[breakable, size=fbox, boxrule=.5pt, pad at break*=1mm, opacityfill=0]
\prompt{Out}{outcolor}{223}{\boxspacing}
\begin{Verbatim}[commandchars=\\\{\}]
        gender            age  hypertension  heart\_disease smoking\_history
count   100000  100000.000000  100000.00000  100000.000000          100000  \textbackslash{}
unique       3            NaN           NaN            NaN               6
top     Female            NaN           NaN            NaN         No Info
freq     58552            NaN           NaN            NaN           35816
mean       NaN      41.885856       0.07485       0.039420             NaN
std        NaN      22.516840       0.26315       0.194593             NaN
min        NaN       0.080000       0.00000       0.000000             NaN
25\%        NaN      24.000000       0.00000       0.000000             NaN
50\%        NaN      43.000000       0.00000       0.000000             NaN
75\%        NaN      60.000000       0.00000       0.000000             NaN
max        NaN      80.000000       1.00000       1.000000             NaN

                  bmi    HbA1c\_level  blood\_glucose\_level       diabetes
count   100000.000000  100000.000000        100000.000000  100000.000000
unique            NaN            NaN                  NaN            NaN
top               NaN            NaN                  NaN            NaN
freq              NaN            NaN                  NaN            NaN
mean        27.320767       5.527507           138.058060       0.085000
std          6.636783       1.070672            40.708136       0.278883
min         10.010000       3.500000            80.000000       0.000000
25\%         23.630000       4.800000           100.000000       0.000000
50\%         27.320000       5.800000           140.000000       0.000000
75\%         29.580000       6.200000           159.000000       0.000000
max         95.690000       9.000000           300.000000       1.000000
\end{Verbatim}
\end{tcolorbox}
        
    This shows, that \texttt{gender} and \texttt{smoking\_history} are
actually categorical columns (they have a few unique values), while
\texttt{hypertension}, \texttt{heart\_disease}, and \texttt{diabetes}
are meant to be boolean, and finally \texttt{age} should be an integer.

    \begin{tcolorbox}[breakable, size=fbox, boxrule=1pt, pad at break*=1mm,colback=cellbackground, colframe=cellborder]
\prompt{In}{incolor}{224}{\boxspacing}
\begin{Verbatim}[commandchars=\\\{\}]
\PY{n}{df}\PY{p}{[}\PY{l+s+s1}{\PYZsq{}}\PY{l+s+s1}{age}\PY{l+s+s1}{\PYZsq{}}\PY{p}{]} \PY{o}{=} \PY{n}{df}\PY{p}{[}\PY{l+s+s1}{\PYZsq{}}\PY{l+s+s1}{age}\PY{l+s+s1}{\PYZsq{}}\PY{p}{]}\PY{o}{.}\PY{n}{astype}\PY{p}{(}\PY{n+nb}{int}\PY{p}{)}
\PY{n}{df}\PY{p}{[}\PY{l+s+s1}{\PYZsq{}}\PY{l+s+s1}{gender}\PY{l+s+s1}{\PYZsq{}}\PY{p}{]} \PY{o}{=} \PY{n}{df}\PY{p}{[}\PY{l+s+s1}{\PYZsq{}}\PY{l+s+s1}{gender}\PY{l+s+s1}{\PYZsq{}}\PY{p}{]}\PY{o}{.}\PY{n}{astype}\PY{p}{(}\PY{l+s+s1}{\PYZsq{}}\PY{l+s+s1}{category}\PY{l+s+s1}{\PYZsq{}}\PY{p}{)}
\PY{n}{df}\PY{p}{[}\PY{l+s+s1}{\PYZsq{}}\PY{l+s+s1}{smoking\PYZus{}history}\PY{l+s+s1}{\PYZsq{}}\PY{p}{]} \PY{o}{=} \PY{n}{df}\PY{p}{[}\PY{l+s+s1}{\PYZsq{}}\PY{l+s+s1}{smoking\PYZus{}history}\PY{l+s+s1}{\PYZsq{}}\PY{p}{]}\PY{o}{.}\PY{n}{astype}\PY{p}{(}\PY{l+s+s1}{\PYZsq{}}\PY{l+s+s1}{category}\PY{l+s+s1}{\PYZsq{}}\PY{p}{)}
\PY{n}{df}\PY{p}{[}\PY{l+s+s1}{\PYZsq{}}\PY{l+s+s1}{hypertension}\PY{l+s+s1}{\PYZsq{}}\PY{p}{]} \PY{o}{=} \PY{n}{df}\PY{p}{[}\PY{l+s+s1}{\PYZsq{}}\PY{l+s+s1}{hypertension}\PY{l+s+s1}{\PYZsq{}}\PY{p}{]}\PY{o}{.}\PY{n}{astype}\PY{p}{(}\PY{n+nb}{bool}\PY{p}{)}
\PY{n}{df}\PY{p}{[}\PY{l+s+s1}{\PYZsq{}}\PY{l+s+s1}{heart\PYZus{}disease}\PY{l+s+s1}{\PYZsq{}}\PY{p}{]} \PY{o}{=} \PY{n}{df}\PY{p}{[}\PY{l+s+s1}{\PYZsq{}}\PY{l+s+s1}{heart\PYZus{}disease}\PY{l+s+s1}{\PYZsq{}}\PY{p}{]}\PY{o}{.}\PY{n}{astype}\PY{p}{(}\PY{n+nb}{bool}\PY{p}{)}
\PY{n}{df}\PY{p}{[}\PY{l+s+s1}{\PYZsq{}}\PY{l+s+s1}{diabetes}\PY{l+s+s1}{\PYZsq{}}\PY{p}{]} \PY{o}{=} \PY{n}{df}\PY{p}{[}\PY{l+s+s1}{\PYZsq{}}\PY{l+s+s1}{diabetes}\PY{l+s+s1}{\PYZsq{}}\PY{p}{]}\PY{o}{.}\PY{n}{astype}\PY{p}{(}\PY{n+nb}{bool}\PY{p}{)}

\PY{n}{df}\PY{o}{.}\PY{n}{dtypes}
\end{Verbatim}
\end{tcolorbox}

            \begin{tcolorbox}[breakable, size=fbox, boxrule=.5pt, pad at break*=1mm, opacityfill=0]
\prompt{Out}{outcolor}{224}{\boxspacing}
\begin{Verbatim}[commandchars=\\\{\}]
gender                 category
age                       int32
hypertension               bool
heart\_disease              bool
smoking\_history        category
bmi                     float64
HbA1c\_level             float64
blood\_glucose\_level       int64
diabetes                   bool
dtype: object
\end{Verbatim}
\end{tcolorbox}
        
    Let's check our categorical values:

    \begin{tcolorbox}[breakable, size=fbox, boxrule=1pt, pad at break*=1mm,colback=cellbackground, colframe=cellborder]
\prompt{In}{incolor}{225}{\boxspacing}
\begin{Verbatim}[commandchars=\\\{\}]
\PY{n}{columns} \PY{o}{=} \PY{n}{df}\PY{o}{.}\PY{n}{select\PYZus{}dtypes}\PY{p}{(}\PY{n}{include}\PY{o}{=}\PY{l+s+s1}{\PYZsq{}}\PY{l+s+s1}{category}\PY{l+s+s1}{\PYZsq{}}\PY{p}{)}\PY{o}{.}\PY{n}{columns}\PY{o}{.}\PY{n}{tolist}\PY{p}{(}\PY{p}{)}

\PY{k}{for} \PY{n}{column} \PY{o+ow}{in} \PY{n}{columns}\PY{p}{:}
    \PY{n}{values} \PY{o}{=} \PY{n+nb}{sorted}\PY{p}{(}\PY{n+nb}{list}\PY{p}{(}\PY{n}{df}\PY{p}{[}\PY{n}{column}\PY{p}{]}\PY{o}{.}\PY{n}{unique}\PY{p}{(}\PY{p}{)}\PY{p}{)}\PY{p}{)}
    \PY{n+nb}{print}\PY{p}{(}\PY{n}{column}\PY{p}{)}
    \PY{n+nb}{print}\PY{p}{(}\PY{n}{values}\PY{p}{)}
    \PY{n+nb}{print}\PY{p}{(}\PY{p}{)}
\end{Verbatim}
\end{tcolorbox}

    \begin{Verbatim}[commandchars=\\\{\}]
gender
['Female', 'Male', 'Other']

smoking\_history
['No Info', 'current', 'ever', 'former', 'never', 'not current']

    \end{Verbatim}

    I would like to fix the inconsistent values in
\texttt{smoking\_history}:

    \begin{tcolorbox}[breakable, size=fbox, boxrule=1pt, pad at break*=1mm,colback=cellbackground, colframe=cellborder]
\prompt{In}{incolor}{226}{\boxspacing}
\begin{Verbatim}[commandchars=\\\{\}]
\PY{n}{df}\PY{p}{[}\PY{l+s+s1}{\PYZsq{}}\PY{l+s+s1}{smoking\PYZus{}history}\PY{l+s+s1}{\PYZsq{}}\PY{p}{]} \PY{o}{=} \PY{n}{df}\PY{p}{[}\PY{l+s+s1}{\PYZsq{}}\PY{l+s+s1}{smoking\PYZus{}history}\PY{l+s+s1}{\PYZsq{}}\PY{p}{]}\PY{o}{.}\PY{n}{replace}\PY{p}{(}\PY{p}{\PYZob{}}\PY{l+s+s1}{\PYZsq{}}\PY{l+s+s1}{No Info}\PY{l+s+s1}{\PYZsq{}}\PY{p}{:} \PY{l+s+s1}{\PYZsq{}}\PY{l+s+s1}{no\PYZus{}info}\PY{l+s+s1}{\PYZsq{}}\PY{p}{,} \PY{l+s+s1}{\PYZsq{}}\PY{l+s+s1}{not current}\PY{l+s+s1}{\PYZsq{}}\PY{p}{:} \PY{l+s+s1}{\PYZsq{}}\PY{l+s+s1}{not\PYZus{}current}\PY{l+s+s1}{\PYZsq{}}\PY{p}{\PYZcb{}}\PY{p}{)}

\PY{n}{columns} \PY{o}{=} \PY{n}{df}\PY{o}{.}\PY{n}{select\PYZus{}dtypes}\PY{p}{(}\PY{n}{include}\PY{o}{=}\PY{l+s+s1}{\PYZsq{}}\PY{l+s+s1}{category}\PY{l+s+s1}{\PYZsq{}}\PY{p}{)}\PY{o}{.}\PY{n}{columns}\PY{o}{.}\PY{n}{tolist}\PY{p}{(}\PY{p}{)}

\PY{k}{for} \PY{n}{column} \PY{o+ow}{in} \PY{n}{columns}\PY{p}{:}
    \PY{n}{values} \PY{o}{=} \PY{n+nb}{sorted}\PY{p}{(}\PY{n+nb}{list}\PY{p}{(}\PY{n}{df}\PY{p}{[}\PY{n}{column}\PY{p}{]}\PY{o}{.}\PY{n}{unique}\PY{p}{(}\PY{p}{)}\PY{p}{)}\PY{p}{)}
    \PY{n+nb}{print}\PY{p}{(}\PY{n}{column}\PY{p}{)}
    \PY{n+nb}{print}\PY{p}{(}\PY{n}{values}\PY{p}{)}
    \PY{n+nb}{print}\PY{p}{(}\PY{p}{)}
\end{Verbatim}
\end{tcolorbox}

    \begin{Verbatim}[commandchars=\\\{\}]
gender
['Female', 'Male', 'Other']

smoking\_history
['current', 'ever', 'former', 'never', 'no\_info', 'not\_current']

    \end{Verbatim}

    \hypertarget{analysing-the-column-values-and-their-distribution}{%
\subsubsection{Analysing the column values and their
distribution}\label{analysing-the-column-values-and-their-distribution}}

When our data comes from a third party, and we have not collected it
ourselves, it is a good idea to check for any inconsistencies in the
values.

    First, I checked that no values were missing.

    \begin{tcolorbox}[breakable, size=fbox, boxrule=1pt, pad at break*=1mm,colback=cellbackground, colframe=cellborder]
\prompt{In}{incolor}{227}{\boxspacing}
\begin{Verbatim}[commandchars=\\\{\}]
\PY{n}{df}\PY{o}{.}\PY{n}{isna}\PY{p}{(}\PY{p}{)}\PY{o}{.}\PY{n}{sum}\PY{p}{(}\PY{p}{)}
\end{Verbatim}
\end{tcolorbox}

            \begin{tcolorbox}[breakable, size=fbox, boxrule=.5pt, pad at break*=1mm, opacityfill=0]
\prompt{Out}{outcolor}{227}{\boxspacing}
\begin{Verbatim}[commandchars=\\\{\}]
gender                 0
age                    0
hypertension           0
heart\_disease          0
smoking\_history        0
bmi                    0
HbA1c\_level            0
blood\_glucose\_level    0
diabetes               0
dtype: int64
\end{Verbatim}
\end{tcolorbox}
        
    If there were missing values, we could drop them with
\texttt{df.dropna(inplace=True)}, but it is not needed.

    \hypertarget{bmi}{%
\paragraph{BMI}\label{bmi}}

The first weirdness I noticed was with BMI.

Let us plot the values of BMI in a histogram:

    \begin{tcolorbox}[breakable, size=fbox, boxrule=1pt, pad at break*=1mm,colback=cellbackground, colframe=cellborder]
\prompt{In}{incolor}{228}{\boxspacing}
\begin{Verbatim}[commandchars=\\\{\}]
\PY{n}{values} \PY{o}{=} \PY{n}{df}\PY{p}{[}\PY{l+s+s1}{\PYZsq{}}\PY{l+s+s1}{bmi}\PY{l+s+s1}{\PYZsq{}}\PY{p}{]}
\PY{n}{counts}\PY{p}{,} \PY{n}{bins}\PY{p}{,} \PY{n}{\PYZus{}} \PY{o}{=} \PY{n}{plt}\PY{o}{.}\PY{n}{hist}\PY{p}{(}\PY{n}{values}\PY{p}{,} \PY{n}{bins}\PY{o}{=}\PY{l+s+s1}{\PYZsq{}}\PY{l+s+s1}{auto}\PY{l+s+s1}{\PYZsq{}}\PY{p}{,} \PY{n}{color}\PY{o}{=}\PY{l+s+s1}{\PYZsq{}}\PY{l+s+s1}{darkgreen}\PY{l+s+s1}{\PYZsq{}}\PY{p}{,} \PY{n}{edgecolor}\PY{o}{=}\PY{l+s+s1}{\PYZsq{}}\PY{l+s+s1}{black}\PY{l+s+s1}{\PYZsq{}}\PY{p}{)}

\PY{n}{plt}\PY{o}{.}\PY{n}{gcf}\PY{p}{(}\PY{p}{)}\PY{o}{.}\PY{n}{set\PYZus{}size\PYZus{}inches}\PY{p}{(}\PY{l+m+mi}{18}\PY{p}{,} \PY{l+m+mi}{6}\PY{p}{)}
\PY{n}{plt}\PY{o}{.}\PY{n}{show}\PY{p}{(}\PY{p}{)}
\end{Verbatim}
\end{tcolorbox}

    \begin{center}
    \adjustimage{max size={0.9\linewidth}{0.9\paperheight}}{diabetes_files/diabetes_22_0.png}
    \end{center}
    { \hspace*{\fill} \\}
    
    That doesn't look good. I kept increasing the bins, before leaving them
auto, but that singular peak will not disappear. Let's try logarithmic
scale, so we can at least see something:

    \begin{tcolorbox}[breakable, size=fbox, boxrule=1pt, pad at break*=1mm,colback=cellbackground, colframe=cellborder]
\prompt{In}{incolor}{229}{\boxspacing}
\begin{Verbatim}[commandchars=\\\{\}]
\PY{n}{values} \PY{o}{=} \PY{n}{df}\PY{p}{[}\PY{l+s+s1}{\PYZsq{}}\PY{l+s+s1}{bmi}\PY{l+s+s1}{\PYZsq{}}\PY{p}{]}
\PY{n}{counts}\PY{p}{,} \PY{n}{bins}\PY{p}{,} \PY{n}{\PYZus{}} \PY{o}{=} \PY{n}{plt}\PY{o}{.}\PY{n}{hist}\PY{p}{(}\PY{n}{values}\PY{p}{,} \PY{n}{bins}\PY{o}{=}\PY{l+s+s1}{\PYZsq{}}\PY{l+s+s1}{auto}\PY{l+s+s1}{\PYZsq{}}\PY{p}{,} \PY{n}{color}\PY{o}{=}\PY{l+s+s1}{\PYZsq{}}\PY{l+s+s1}{darkgreen}\PY{l+s+s1}{\PYZsq{}}\PY{p}{,} \PY{n}{edgecolor}\PY{o}{=}\PY{l+s+s1}{\PYZsq{}}\PY{l+s+s1}{black}\PY{l+s+s1}{\PYZsq{}}\PY{p}{)}
\PY{n}{plt}\PY{o}{.}\PY{n}{yscale}\PY{p}{(}\PY{l+s+s1}{\PYZsq{}}\PY{l+s+s1}{log}\PY{l+s+s1}{\PYZsq{}}\PY{p}{)}
\PY{n}{plt}\PY{o}{.}\PY{n}{gcf}\PY{p}{(}\PY{p}{)}\PY{o}{.}\PY{n}{set\PYZus{}size\PYZus{}inches}\PY{p}{(}\PY{l+m+mi}{18}\PY{p}{,} \PY{l+m+mi}{6}\PY{p}{)}
\PY{n}{plt}\PY{o}{.}\PY{n}{show}\PY{p}{(}\PY{p}{)}
\end{Verbatim}
\end{tcolorbox}

    \begin{center}
    \adjustimage{max size={0.9\linewidth}{0.9\paperheight}}{diabetes_files/diabetes_24_0.png}
    \end{center}
    { \hspace*{\fill} \\}
    
    Is that a single value of BMI? Let's count the most common values:

    \begin{tcolorbox}[breakable, size=fbox, boxrule=1pt, pad at break*=1mm,colback=cellbackground, colframe=cellborder]
\prompt{In}{incolor}{230}{\boxspacing}
\begin{Verbatim}[commandchars=\\\{\}]
\PY{n}{max\PYZus{}count\PYZus{}bin} \PY{o}{=} \PY{n}{np}\PY{o}{.}\PY{n}{argmax}\PY{p}{(}\PY{n}{counts}\PY{p}{)}
\PY{n}{max\PYZus{}count} \PY{o}{=} \PY{n}{counts}\PY{p}{[}\PY{n}{max\PYZus{}count\PYZus{}bin}\PY{p}{]}
\PY{n}{max\PYZus{}bin\PYZus{}range} \PY{o}{=} \PY{p}{(}\PY{n}{bins}\PY{p}{[}\PY{n}{max\PYZus{}count\PYZus{}bin}\PY{p}{]}\PY{p}{,} \PY{n}{bins}\PY{p}{[}\PY{n}{max\PYZus{}count\PYZus{}bin} \PY{o}{+} \PY{l+m+mi}{1}\PY{p}{]}\PY{p}{)}

\PY{n}{max\PYZus{}bin\PYZus{}range}

\PY{n}{unique\PYZus{}values}\PY{p}{,} \PY{n}{counts} \PY{o}{=} \PY{n}{np}\PY{o}{.}\PY{n}{unique}\PY{p}{(}\PY{n}{values}\PY{p}{,} \PY{n}{return\PYZus{}counts}\PY{o}{=}\PY{k+kc}{True}\PY{p}{)}
\PY{n}{sorted\PYZus{}indices} \PY{o}{=} \PY{n}{np}\PY{o}{.}\PY{n}{argsort}\PY{p}{(}\PY{n}{counts}\PY{p}{)}\PY{p}{[}\PY{p}{:}\PY{p}{:}\PY{o}{\PYZhy{}}\PY{l+m+mi}{1}\PY{p}{]}
\PY{n}{sorted\PYZus{}values} \PY{o}{=} \PY{n}{unique\PYZus{}values}\PY{p}{[}\PY{n}{sorted\PYZus{}indices}\PY{p}{]}
\PY{n}{sorted\PYZus{}counts} \PY{o}{=} \PY{n}{counts}\PY{p}{[}\PY{n}{sorted\PYZus{}indices}\PY{p}{]}

\PY{n}{top\PYZus{}10\PYZus{}values} \PY{o}{=} \PY{n}{sorted\PYZus{}values}\PY{p}{[}\PY{p}{:}\PY{l+m+mi}{10}\PY{p}{]}
\PY{n}{top\PYZus{}10\PYZus{}counts} \PY{o}{=} \PY{n}{sorted\PYZus{}counts}\PY{p}{[}\PY{p}{:}\PY{l+m+mi}{10}\PY{p}{]}

\PY{n+nb}{print}\PY{p}{(}\PY{n}{top\PYZus{}10\PYZus{}values}\PY{p}{)}
\PY{n+nb}{print}\PY{p}{(}\PY{n}{top\PYZus{}10\PYZus{}counts}\PY{p}{)}
\end{Verbatim}
\end{tcolorbox}

    \begin{Verbatim}[commandchars=\\\{\}]
[27.32 23.   27.12 27.8  24.96 22.4  25.   25.6  26.7  24.5 ]
[25495   103   101   100   100    99    99    98    94    94]
    \end{Verbatim}

    Yes, for some reason, we have over \(\frac{1}{4}\) entries of the
dataset with the exact BMI of 27.32. Let's look at these:

    \begin{tcolorbox}[breakable, size=fbox, boxrule=1pt, pad at break*=1mm,colback=cellbackground, colframe=cellborder]
\prompt{In}{incolor}{231}{\boxspacing}
\begin{Verbatim}[commandchars=\\\{\}]
\PY{n}{weird\PYZus{}entries} \PY{o}{=} \PY{n}{df}\PY{p}{[}\PY{n}{df}\PY{p}{[}\PY{l+s+s1}{\PYZsq{}}\PY{l+s+s1}{bmi}\PY{l+s+s1}{\PYZsq{}}\PY{p}{]} \PY{o}{==} \PY{l+m+mf}{27.32}\PY{p}{]}
\PY{n}{weird\PYZus{}entries}\PY{o}{.}\PY{n}{head}\PY{p}{(}\PY{l+m+mi}{10}\PY{p}{)}
\end{Verbatim}
\end{tcolorbox}

            \begin{tcolorbox}[breakable, size=fbox, boxrule=.5pt, pad at break*=1mm, opacityfill=0]
\prompt{Out}{outcolor}{231}{\boxspacing}
\begin{Verbatim}[commandchars=\\\{\}]
    gender  age  hypertension  heart\_disease smoking\_history    bmi
1   Female   54         False          False         no\_info  27.32  \textbackslash{}
2     Male   28         False          False           never  27.32
5   Female   20         False          False           never  27.32
9   Female   32         False          False           never  27.32
10  Female   53         False          False           never  27.32
14  Female   76         False          False         no\_info  27.32
15    Male   78         False          False         no\_info  27.32
18  Female   42         False          False         no\_info  27.32
26    Male   67         False           True     not\_current  27.32
38    Male   50          True          False         current  27.32

    HbA1c\_level  blood\_glucose\_level  diabetes
1           6.6                   80     False
2           5.7                  158     False
5           6.6                   85     False
9           5.0                  100     False
10          6.1                   85     False
14          5.0                  160     False
15          6.6                  126     False
18          5.7                   80     False
26          6.5                  200      True
38          5.7                  260      True
\end{Verbatim}
\end{tcolorbox}
        
    These all seem like real entries, not duplicates of the same entry. My
best guess for what happened, is that when the BMI was missing, they put
down the mean of the rest of the dataset, which is exactly
\texttt{27.32}.

    \begin{tcolorbox}[breakable, size=fbox, boxrule=1pt, pad at break*=1mm,colback=cellbackground, colframe=cellborder]
\prompt{In}{incolor}{232}{\boxspacing}
\begin{Verbatim}[commandchars=\\\{\}]
\PY{n}{df}\PY{p}{[}\PY{n}{df}\PY{p}{[}\PY{l+s+s1}{\PYZsq{}}\PY{l+s+s1}{bmi}\PY{l+s+s1}{\PYZsq{}}\PY{p}{]} \PY{o}{!=} \PY{l+m+mf}{27.32}\PY{p}{]}\PY{p}{[}\PY{l+s+s1}{\PYZsq{}}\PY{l+s+s1}{bmi}\PY{l+s+s1}{\PYZsq{}}\PY{p}{]}\PY{o}{.}\PY{n}{mean}\PY{p}{(}\PY{p}{)}
\end{Verbatim}
\end{tcolorbox}

            \begin{tcolorbox}[breakable, size=fbox, boxrule=.5pt, pad at break*=1mm, opacityfill=0]
\prompt{Out}{outcolor}{232}{\boxspacing}
\begin{Verbatim}[commandchars=\\\{\}]
27.321029595329176
\end{Verbatim}
\end{tcolorbox}
        
    In a real life scenario, I would clarify this with the person who gave
me the data, however for this homework project, I will assume that when
a person looks average weight, they may not bother measuring their BMI,
so the average is probably a good approximation, so I will accept this.

Another concern I had was with extreme cases of BMI, such as the values
80 and above. I choose to keep these, because they are not
overrepresented and diabetes is linked to extreme obesity, so these are
important entry points for prediction.

    \hypertarget{age}{%
\paragraph{Age}\label{age}}

Next issue is with the age.

Let's plot the values:

    \begin{tcolorbox}[breakable, size=fbox, boxrule=1pt, pad at break*=1mm,colback=cellbackground, colframe=cellborder]
\prompt{In}{incolor}{233}{\boxspacing}
\begin{Verbatim}[commandchars=\\\{\}]
\PY{n}{values} \PY{o}{=} \PY{n}{df}\PY{p}{[}\PY{l+s+s1}{\PYZsq{}}\PY{l+s+s1}{age}\PY{l+s+s1}{\PYZsq{}}\PY{p}{]}
\PY{n}{num\PYZus{}bins} \PY{o}{=} \PY{n}{values}\PY{o}{.}\PY{n}{nunique}\PY{p}{(}\PY{p}{)}
\PY{n}{counts}\PY{p}{,} \PY{n}{bins}\PY{p}{,} \PY{n}{\PYZus{}} \PY{o}{=} \PY{n}{plt}\PY{o}{.}\PY{n}{hist}\PY{p}{(}\PY{n}{values}\PY{p}{,} \PY{n}{bins}\PY{o}{=}\PY{n}{num\PYZus{}bins}\PY{p}{,} \PY{n}{color}\PY{o}{=}\PY{l+s+s1}{\PYZsq{}}\PY{l+s+s1}{darkgreen}\PY{l+s+s1}{\PYZsq{}}\PY{p}{,} \PY{n}{edgecolor}\PY{o}{=}\PY{l+s+s1}{\PYZsq{}}\PY{l+s+s1}{black}\PY{l+s+s1}{\PYZsq{}}\PY{p}{)}

\PY{n}{plt}\PY{o}{.}\PY{n}{gcf}\PY{p}{(}\PY{p}{)}\PY{o}{.}\PY{n}{set\PYZus{}size\PYZus{}inches}\PY{p}{(}\PY{l+m+mi}{18}\PY{p}{,} \PY{l+m+mi}{6}\PY{p}{)}
\PY{n}{plt}\PY{o}{.}\PY{n}{show}\PY{p}{(}\PY{p}{)}
\end{Verbatim}
\end{tcolorbox}

    \begin{center}
    \adjustimage{max size={0.9\linewidth}{0.9\paperheight}}{diabetes_files/diabetes_33_0.png}
    \end{center}
    { \hspace*{\fill} \\}
    
    We have a lot of people aged \texttt{80}. It looks like older people are
overrepresented in this data and probably not all of them are exactly
\texttt{80}, they just cut off the age to fit into a \texttt{{[}0,80{]}}
interval.

    \begin{tcolorbox}[breakable, size=fbox, boxrule=1pt, pad at break*=1mm,colback=cellbackground, colframe=cellborder]
\prompt{In}{incolor}{234}{\boxspacing}
\begin{Verbatim}[commandchars=\\\{\}]
\PY{n}{df}\PY{p}{[}\PY{n}{df}\PY{p}{[}\PY{l+s+s1}{\PYZsq{}}\PY{l+s+s1}{age}\PY{l+s+s1}{\PYZsq{}}\PY{p}{]}\PY{o}{==}\PY{l+m+mi}{80}\PY{p}{]}\PY{p}{[}\PY{l+s+s1}{\PYZsq{}}\PY{l+s+s1}{age}\PY{l+s+s1}{\PYZsq{}}\PY{p}{]}\PY{o}{.}\PY{n}{count}\PY{p}{(}\PY{p}{)}
\end{Verbatim}
\end{tcolorbox}

            \begin{tcolorbox}[breakable, size=fbox, boxrule=.5pt, pad at break*=1mm, opacityfill=0]
\prompt{Out}{outcolor}{234}{\boxspacing}
\begin{Verbatim}[commandchars=\\\{\}]
5621
\end{Verbatim}
\end{tcolorbox}
        
    In this case, I choose to remove these values, because I feel like this
could cause some skewing.

    \begin{tcolorbox}[breakable, size=fbox, boxrule=1pt, pad at break*=1mm,colback=cellbackground, colframe=cellborder]
\prompt{In}{incolor}{235}{\boxspacing}
\begin{Verbatim}[commandchars=\\\{\}]
\PY{n}{df} \PY{o}{=} \PY{n}{df}\PY{p}{[}\PY{n}{df}\PY{p}{[}\PY{l+s+s1}{\PYZsq{}}\PY{l+s+s1}{age}\PY{l+s+s1}{\PYZsq{}}\PY{p}{]} \PY{o}{\PYZlt{}} \PY{l+m+mi}{80}\PY{p}{]}
\end{Verbatim}
\end{tcolorbox}

    \hypertarget{gender}{%
\paragraph{Gender}\label{gender}}

Let's plot the values:

    \begin{tcolorbox}[breakable, size=fbox, boxrule=1pt, pad at break*=1mm,colback=cellbackground, colframe=cellborder]
\prompt{In}{incolor}{236}{\boxspacing}
\begin{Verbatim}[commandchars=\\\{\}]
\PY{n}{values} \PY{o}{=} \PY{n}{df}\PY{p}{[}\PY{l+s+s1}{\PYZsq{}}\PY{l+s+s1}{gender}\PY{l+s+s1}{\PYZsq{}}\PY{p}{]}
\PY{n}{num\PYZus{}bins} \PY{o}{=} \PY{n}{values}\PY{o}{.}\PY{n}{nunique}\PY{p}{(}\PY{p}{)}
\PY{n}{counts}\PY{p}{,} \PY{n}{bins}\PY{p}{,} \PY{n}{\PYZus{}} \PY{o}{=} \PY{n}{plt}\PY{o}{.}\PY{n}{hist}\PY{p}{(}\PY{n}{values}\PY{p}{,} \PY{n}{bins}\PY{o}{=}\PY{n}{num\PYZus{}bins}\PY{p}{,} \PY{n}{color}\PY{o}{=}\PY{l+s+s1}{\PYZsq{}}\PY{l+s+s1}{darkgreen}\PY{l+s+s1}{\PYZsq{}}\PY{p}{,} \PY{n}{edgecolor}\PY{o}{=}\PY{l+s+s1}{\PYZsq{}}\PY{l+s+s1}{black}\PY{l+s+s1}{\PYZsq{}}\PY{p}{)}

\PY{n}{plt}\PY{o}{.}\PY{n}{gcf}\PY{p}{(}\PY{p}{)}\PY{o}{.}\PY{n}{set\PYZus{}size\PYZus{}inches}\PY{p}{(}\PY{l+m+mi}{3}\PY{p}{,} \PY{l+m+mi}{3}\PY{p}{)}
\PY{n}{plt}\PY{o}{.}\PY{n}{show}\PY{p}{(}\PY{p}{)}
\end{Verbatim}
\end{tcolorbox}

    \begin{center}
    \adjustimage{max size={0.9\linewidth}{0.9\paperheight}}{diabetes_files/diabetes_39_0.png}
    \end{center}
    { \hspace*{\fill} \\}
    
    The category Other can mean many things. For medical research, we are
concerned about the biological sex of the participants, so I will remove
these entries too.

    \begin{tcolorbox}[breakable, size=fbox, boxrule=1pt, pad at break*=1mm,colback=cellbackground, colframe=cellborder]
\prompt{In}{incolor}{237}{\boxspacing}
\begin{Verbatim}[commandchars=\\\{\}]
\PY{n}{df} \PY{o}{=} \PY{n}{df}\PY{p}{[}\PY{n}{df}\PY{p}{[}\PY{l+s+s1}{\PYZsq{}}\PY{l+s+s1}{gender}\PY{l+s+s1}{\PYZsq{}}\PY{p}{]} \PY{o}{!=} \PY{l+s+s1}{\PYZsq{}}\PY{l+s+s1}{Other}\PY{l+s+s1}{\PYZsq{}}\PY{p}{]}
\PY{n}{df}\PY{p}{[}\PY{l+s+s1}{\PYZsq{}}\PY{l+s+s1}{gender}\PY{l+s+s1}{\PYZsq{}}\PY{p}{]} \PY{o}{=} \PY{n}{df}\PY{p}{[}\PY{l+s+s1}{\PYZsq{}}\PY{l+s+s1}{gender}\PY{l+s+s1}{\PYZsq{}}\PY{p}{]}\PY{o}{.}\PY{n}{cat}\PY{o}{.}\PY{n}{remove\PYZus{}categories}\PY{p}{(}\PY{l+s+s1}{\PYZsq{}}\PY{l+s+s1}{Other}\PY{l+s+s1}{\PYZsq{}}\PY{p}{)}
\PY{n}{df}\PY{p}{[}\PY{l+s+s1}{\PYZsq{}}\PY{l+s+s1}{gender}\PY{l+s+s1}{\PYZsq{}}\PY{p}{]}\PY{o}{.}\PY{n}{unique}\PY{p}{(}\PY{p}{)}
\end{Verbatim}
\end{tcolorbox}

            \begin{tcolorbox}[breakable, size=fbox, boxrule=.5pt, pad at break*=1mm, opacityfill=0]
\prompt{Out}{outcolor}{237}{\boxspacing}
\begin{Verbatim}[commandchars=\\\{\}]
['Female', 'Male']
Categories (2, object): ['Female', 'Male']
\end{Verbatim}
\end{tcolorbox}
        
    \hypertarget{blood-glucose-level}{%
\paragraph{Blood glucose level}\label{blood-glucose-level}}

    \begin{tcolorbox}[breakable, size=fbox, boxrule=1pt, pad at break*=1mm,colback=cellbackground, colframe=cellborder]
\prompt{In}{incolor}{238}{\boxspacing}
\begin{Verbatim}[commandchars=\\\{\}]
\PY{n}{values} \PY{o}{=} \PY{n}{df}\PY{p}{[}\PY{l+s+s1}{\PYZsq{}}\PY{l+s+s1}{blood\PYZus{}glucose\PYZus{}level}\PY{l+s+s1}{\PYZsq{}}\PY{p}{]}
\PY{n}{counts}\PY{p}{,} \PY{n}{bins}\PY{p}{,} \PY{n}{\PYZus{}} \PY{o}{=} \PY{n}{plt}\PY{o}{.}\PY{n}{hist}\PY{p}{(}\PY{n}{values}\PY{p}{,} \PY{n}{bins}\PY{o}{=}\PY{l+m+mi}{200}\PY{p}{,} \PY{n}{color}\PY{o}{=}\PY{l+s+s1}{\PYZsq{}}\PY{l+s+s1}{darkgreen}\PY{l+s+s1}{\PYZsq{}}\PY{p}{,} \PY{n}{edgecolor}\PY{o}{=}\PY{l+s+s1}{\PYZsq{}}\PY{l+s+s1}{black}\PY{l+s+s1}{\PYZsq{}}\PY{p}{)}

\PY{n}{plt}\PY{o}{.}\PY{n}{gcf}\PY{p}{(}\PY{p}{)}\PY{o}{.}\PY{n}{set\PYZus{}size\PYZus{}inches}\PY{p}{(}\PY{l+m+mi}{18}\PY{p}{,} \PY{l+m+mi}{4}\PY{p}{)}
\PY{n}{plt}\PY{o}{.}\PY{n}{show}\PY{p}{(}\PY{p}{)}
\end{Verbatim}
\end{tcolorbox}

    \begin{center}
    \adjustimage{max size={0.9\linewidth}{0.9\paperheight}}{diabetes_files/diabetes_43_0.png}
    \end{center}
    { \hspace*{\fill} \\}
    
    It seems like this is not a continuous spectrum. There is also a
concentration of values around \(155-160\). I'm not sure how to
interpret this. The values are probably in \texttt{mg/dl}. For fasting
levels, \(80-100\) is normal, \(101-125\) is elevated and above \(126\)
is high. Other than that, I would need to ask someone with medical
knowledge on why this is happening. For now, I have to accept these, as
there is no clear way on how to fix them or if they need to be fixed at
all.

    \hypertarget{blood-sugar-level}{%
\paragraph{Blood sugar level}\label{blood-sugar-level}}

    \begin{tcolorbox}[breakable, size=fbox, boxrule=1pt, pad at break*=1mm,colback=cellbackground, colframe=cellborder]
\prompt{In}{incolor}{239}{\boxspacing}
\begin{Verbatim}[commandchars=\\\{\}]
\PY{n}{values} \PY{o}{=} \PY{n}{df}\PY{p}{[}\PY{l+s+s1}{\PYZsq{}}\PY{l+s+s1}{HbA1c\PYZus{}level}\PY{l+s+s1}{\PYZsq{}}\PY{p}{]}
\PY{n}{counts}\PY{p}{,} \PY{n}{bins}\PY{p}{,} \PY{n}{\PYZus{}} \PY{o}{=} \PY{n}{plt}\PY{o}{.}\PY{n}{hist}\PY{p}{(}\PY{n}{values}\PY{p}{,} \PY{n}{bins}\PY{o}{=}\PY{l+s+s1}{\PYZsq{}}\PY{l+s+s1}{auto}\PY{l+s+s1}{\PYZsq{}}\PY{p}{,} \PY{n}{color}\PY{o}{=}\PY{l+s+s1}{\PYZsq{}}\PY{l+s+s1}{darkgreen}\PY{l+s+s1}{\PYZsq{}}\PY{p}{,} \PY{n}{edgecolor}\PY{o}{=}\PY{l+s+s1}{\PYZsq{}}\PY{l+s+s1}{black}\PY{l+s+s1}{\PYZsq{}}\PY{p}{)}

\PY{n}{plt}\PY{o}{.}\PY{n}{gcf}\PY{p}{(}\PY{p}{)}\PY{o}{.}\PY{n}{set\PYZus{}size\PYZus{}inches}\PY{p}{(}\PY{l+m+mi}{18}\PY{p}{,} \PY{l+m+mi}{4}\PY{p}{)}
\PY{n}{plt}\PY{o}{.}\PY{n}{show}\PY{p}{(}\PY{p}{)}
\end{Verbatim}
\end{tcolorbox}

    \begin{center}
    \adjustimage{max size={0.9\linewidth}{0.9\paperheight}}{diabetes_files/diabetes_46_0.png}
    \end{center}
    { \hspace*{\fill} \\}
    
    For \texttt{HbA1c}, less than \(5.6\%\) is normal, between
\(5.7\% - 6.4\%\) is elevated (prediabetes) and above \(6.5\%\) is high
(diabetes).

Although increasing the granuality does reveal that not every value is
represented, but the precision is only one digit after the integer, so I
think this looks okay.

    \hypertarget{smoking-history}{%
\paragraph{Smoking history}\label{smoking-history}}

    \begin{tcolorbox}[breakable, size=fbox, boxrule=1pt, pad at break*=1mm,colback=cellbackground, colframe=cellborder]
\prompt{In}{incolor}{240}{\boxspacing}
\begin{Verbatim}[commandchars=\\\{\}]
\PY{n}{values} \PY{o}{=} \PY{n}{df}\PY{p}{[}\PY{l+s+s1}{\PYZsq{}}\PY{l+s+s1}{smoking\PYZus{}history}\PY{l+s+s1}{\PYZsq{}}\PY{p}{]}
\PY{n}{num\PYZus{}bins} \PY{o}{=} \PY{n}{values}\PY{o}{.}\PY{n}{nunique}\PY{p}{(}\PY{p}{)}
\PY{n}{counts}\PY{p}{,} \PY{n}{bins}\PY{p}{,} \PY{n}{\PYZus{}} \PY{o}{=} \PY{n}{plt}\PY{o}{.}\PY{n}{hist}\PY{p}{(}\PY{n}{values}\PY{p}{,} \PY{n}{bins}\PY{o}{=}\PY{n}{num\PYZus{}bins}\PY{p}{,} \PY{n}{color}\PY{o}{=}\PY{l+s+s1}{\PYZsq{}}\PY{l+s+s1}{darkgreen}\PY{l+s+s1}{\PYZsq{}}\PY{p}{,} \PY{n}{edgecolor}\PY{o}{=}\PY{l+s+s1}{\PYZsq{}}\PY{l+s+s1}{black}\PY{l+s+s1}{\PYZsq{}}\PY{p}{)}

\PY{n}{plt}\PY{o}{.}\PY{n}{gcf}\PY{p}{(}\PY{p}{)}\PY{o}{.}\PY{n}{set\PYZus{}size\PYZus{}inches}\PY{p}{(}\PY{l+m+mi}{10}\PY{p}{,} \PY{l+m+mi}{4}\PY{p}{)}
\PY{n}{plt}\PY{o}{.}\PY{n}{show}\PY{p}{(}\PY{p}{)}
\end{Verbatim}
\end{tcolorbox}

    \begin{center}
    \adjustimage{max size={0.9\linewidth}{0.9\paperheight}}{diabetes_files/diabetes_49_0.png}
    \end{center}
    { \hspace*{\fill} \\}
    
    We have a large amount of entries with \texttt{no\_info} as a value. I
was debating whether to remove these, or not, but I ended up keeping
them, because they represent a large chunk of our data.

I was also questioning what \texttt{ever} means, given the other
possibilities present, I believe it means they have tried smoking in the
past, but weren't addicted.

    \hypertarget{hypertension}{%
\paragraph{Hypertension}\label{hypertension}}

    \begin{tcolorbox}[breakable, size=fbox, boxrule=1pt, pad at break*=1mm,colback=cellbackground, colframe=cellborder]
\prompt{In}{incolor}{241}{\boxspacing}
\begin{Verbatim}[commandchars=\\\{\}]
\PY{n}{df}\PY{p}{[}\PY{l+s+s1}{\PYZsq{}}\PY{l+s+s1}{hypertension}\PY{l+s+s1}{\PYZsq{}}\PY{p}{]}\PY{o}{.}\PY{n}{value\PYZus{}counts}\PY{p}{(}\PY{p}{)}

\PY{n}{value\PYZus{}counts} \PY{o}{=} \PY{n}{df}\PY{p}{[}\PY{l+s+s1}{\PYZsq{}}\PY{l+s+s1}{hypertension}\PY{l+s+s1}{\PYZsq{}}\PY{p}{]}\PY{o}{.}\PY{n}{value\PYZus{}counts}\PY{p}{(}\PY{p}{)}

\PY{n}{plt}\PY{o}{.}\PY{n}{figure}\PY{p}{(}\PY{n}{figsize}\PY{o}{=}\PY{p}{(}\PY{l+m+mi}{3}\PY{p}{,}\PY{l+m+mi}{3}\PY{p}{)}\PY{p}{)}
\PY{n}{plt}\PY{o}{.}\PY{n}{bar}\PY{p}{(}\PY{n}{value\PYZus{}counts}\PY{o}{.}\PY{n}{index}\PY{p}{,} \PY{n}{value\PYZus{}counts}\PY{o}{.}\PY{n}{values}\PY{p}{,} \PY{n}{color}\PY{o}{=}\PY{l+s+s1}{\PYZsq{}}\PY{l+s+s1}{darkgreen}\PY{l+s+s1}{\PYZsq{}}\PY{p}{,} \PY{n}{edgecolor}\PY{o}{=}\PY{l+s+s1}{\PYZsq{}}\PY{l+s+s1}{black}\PY{l+s+s1}{\PYZsq{}}\PY{p}{)}
\PY{n}{plt}\PY{o}{.}\PY{n}{xticks}\PY{p}{(}\PY{n}{ticks}\PY{o}{=}\PY{p}{[}\PY{l+m+mi}{0}\PY{p}{,}\PY{l+m+mi}{1}\PY{p}{]}\PY{p}{,} \PY{n}{labels}\PY{o}{=}\PY{p}{[}\PY{l+s+s1}{\PYZsq{}}\PY{l+s+s1}{False}\PY{l+s+s1}{\PYZsq{}}\PY{p}{,} \PY{l+s+s1}{\PYZsq{}}\PY{l+s+s1}{True}\PY{l+s+s1}{\PYZsq{}}\PY{p}{]}\PY{p}{)} 
\PY{n}{plt}\PY{o}{.}\PY{n}{show}\PY{p}{(}\PY{p}{)}
\end{Verbatim}
\end{tcolorbox}

    \begin{center}
    \adjustimage{max size={0.9\linewidth}{0.9\paperheight}}{diabetes_files/diabetes_52_0.png}
    \end{center}
    { \hspace*{\fill} \\}
    
    It seems like a reasonable distribution.

    \hypertarget{heart-disease}{%
\paragraph{Heart disease}\label{heart-disease}}

    \begin{tcolorbox}[breakable, size=fbox, boxrule=1pt, pad at break*=1mm,colback=cellbackground, colframe=cellborder]
\prompt{In}{incolor}{242}{\boxspacing}
\begin{Verbatim}[commandchars=\\\{\}]
\PY{n}{df}\PY{p}{[}\PY{l+s+s1}{\PYZsq{}}\PY{l+s+s1}{heart\PYZus{}disease}\PY{l+s+s1}{\PYZsq{}}\PY{p}{]}\PY{o}{.}\PY{n}{value\PYZus{}counts}\PY{p}{(}\PY{p}{)}

\PY{n}{value\PYZus{}counts} \PY{o}{=} \PY{n}{df}\PY{p}{[}\PY{l+s+s1}{\PYZsq{}}\PY{l+s+s1}{hypertension}\PY{l+s+s1}{\PYZsq{}}\PY{p}{]}\PY{o}{.}\PY{n}{value\PYZus{}counts}\PY{p}{(}\PY{p}{)}

\PY{n}{plt}\PY{o}{.}\PY{n}{figure}\PY{p}{(}\PY{n}{figsize}\PY{o}{=}\PY{p}{(}\PY{l+m+mi}{3}\PY{p}{,}\PY{l+m+mi}{3}\PY{p}{)}\PY{p}{)}
\PY{n}{plt}\PY{o}{.}\PY{n}{bar}\PY{p}{(}\PY{n}{value\PYZus{}counts}\PY{o}{.}\PY{n}{index}\PY{p}{,} \PY{n}{value\PYZus{}counts}\PY{o}{.}\PY{n}{values}\PY{p}{,} \PY{n}{color}\PY{o}{=}\PY{l+s+s1}{\PYZsq{}}\PY{l+s+s1}{darkgreen}\PY{l+s+s1}{\PYZsq{}}\PY{p}{,} \PY{n}{edgecolor}\PY{o}{=}\PY{l+s+s1}{\PYZsq{}}\PY{l+s+s1}{black}\PY{l+s+s1}{\PYZsq{}}\PY{p}{)}
\PY{n}{plt}\PY{o}{.}\PY{n}{xticks}\PY{p}{(}\PY{n}{ticks}\PY{o}{=}\PY{p}{[}\PY{l+m+mi}{0}\PY{p}{,}\PY{l+m+mi}{1}\PY{p}{]}\PY{p}{,} \PY{n}{labels}\PY{o}{=}\PY{p}{[}\PY{l+s+s1}{\PYZsq{}}\PY{l+s+s1}{False}\PY{l+s+s1}{\PYZsq{}}\PY{p}{,} \PY{l+s+s1}{\PYZsq{}}\PY{l+s+s1}{True}\PY{l+s+s1}{\PYZsq{}}\PY{p}{]}\PY{p}{)} 
\PY{n}{plt}\PY{o}{.}\PY{n}{show}\PY{p}{(}\PY{p}{)}
\end{Verbatim}
\end{tcolorbox}

    \begin{center}
    \adjustimage{max size={0.9\linewidth}{0.9\paperheight}}{diabetes_files/diabetes_55_0.png}
    \end{center}
    { \hspace*{\fill} \\}
    
    Similarly, this looks okay.

    \hypertarget{diabetes}{%
\paragraph{Diabetes}\label{diabetes}}

Diabetes is our target value.

    \begin{tcolorbox}[breakable, size=fbox, boxrule=1pt, pad at break*=1mm,colback=cellbackground, colframe=cellborder]
\prompt{In}{incolor}{243}{\boxspacing}
\begin{Verbatim}[commandchars=\\\{\}]
\PY{n}{df}\PY{p}{[}\PY{l+s+s1}{\PYZsq{}}\PY{l+s+s1}{diabetes}\PY{l+s+s1}{\PYZsq{}}\PY{p}{]}\PY{o}{.}\PY{n}{value\PYZus{}counts}\PY{p}{(}\PY{p}{)}

\PY{n}{value\PYZus{}counts} \PY{o}{=} \PY{n}{df}\PY{p}{[}\PY{l+s+s1}{\PYZsq{}}\PY{l+s+s1}{hypertension}\PY{l+s+s1}{\PYZsq{}}\PY{p}{]}\PY{o}{.}\PY{n}{value\PYZus{}counts}\PY{p}{(}\PY{p}{)}

\PY{n}{plt}\PY{o}{.}\PY{n}{figure}\PY{p}{(}\PY{n}{figsize}\PY{o}{=}\PY{p}{(}\PY{l+m+mi}{10}\PY{p}{,}\PY{l+m+mi}{3}\PY{p}{)}\PY{p}{)}
\PY{n}{plt}\PY{o}{.}\PY{n}{bar}\PY{p}{(}\PY{n}{value\PYZus{}counts}\PY{o}{.}\PY{n}{index}\PY{p}{,} \PY{n}{value\PYZus{}counts}\PY{o}{.}\PY{n}{values}\PY{p}{,} \PY{n}{color}\PY{o}{=}\PY{l+s+s1}{\PYZsq{}}\PY{l+s+s1}{darkgreen}\PY{l+s+s1}{\PYZsq{}}\PY{p}{,} \PY{n}{edgecolor}\PY{o}{=}\PY{l+s+s1}{\PYZsq{}}\PY{l+s+s1}{black}\PY{l+s+s1}{\PYZsq{}}\PY{p}{)}
\PY{n}{plt}\PY{o}{.}\PY{n}{xticks}\PY{p}{(}\PY{n}{ticks}\PY{o}{=}\PY{p}{[}\PY{l+m+mi}{0}\PY{p}{,}\PY{l+m+mi}{1}\PY{p}{]}\PY{p}{,} \PY{n}{labels}\PY{o}{=}\PY{p}{[}\PY{l+s+s1}{\PYZsq{}}\PY{l+s+s1}{False}\PY{l+s+s1}{\PYZsq{}}\PY{p}{,} \PY{l+s+s1}{\PYZsq{}}\PY{l+s+s1}{True}\PY{l+s+s1}{\PYZsq{}}\PY{p}{]}\PY{p}{)} 
\PY{n}{plt}\PY{o}{.}\PY{n}{show}\PY{p}{(}\PY{p}{)}
\end{Verbatim}
\end{tcolorbox}

    \begin{center}
    \adjustimage{max size={0.9\linewidth}{0.9\paperheight}}{diabetes_files/diabetes_58_0.png}
    \end{center}
    { \hspace*{\fill} \\}
    
    This seems like a good ratio, similar to the generic population.

    \hypertarget{analysis}{%
\section{Analysis}\label{analysis}}

    \hypertarget{complete-multivariate-linear-regression}{%
\subsection{Complete multivariate linear
regression}\label{complete-multivariate-linear-regression}}

The first thing I will try is create a multivariate linear regression
model, using all of the available variables.

In order to do this, we must convert the categorical variables into 0/1
variables. This can be done with the \texttt{get\_dummies} function, as
seen below.

    \begin{tcolorbox}[breakable, size=fbox, boxrule=1pt, pad at break*=1mm,colback=cellbackground, colframe=cellborder]
\prompt{In}{incolor}{244}{\boxspacing}
\begin{Verbatim}[commandchars=\\\{\}]
\PY{n}{df\PYZus{}encoded} \PY{o}{=} \PY{n}{pd}\PY{o}{.}\PY{n}{get\PYZus{}dummies}\PY{p}{(}\PY{n}{df}\PY{p}{)}
\PY{n}{df\PYZus{}encoded}\PY{o}{.}\PY{n}{head}\PY{p}{(}\PY{p}{)}
\end{Verbatim}
\end{tcolorbox}

            \begin{tcolorbox}[breakable, size=fbox, boxrule=.5pt, pad at break*=1mm, opacityfill=0]
\prompt{Out}{outcolor}{244}{\boxspacing}
\begin{Verbatim}[commandchars=\\\{\}]
   age  hypertension  heart\_disease    bmi  HbA1c\_level  blood\_glucose\_level
1   54         False          False  27.32          6.6                   80  \textbackslash{}
2   28         False          False  27.32          5.7                  158
3   36         False          False  23.45          5.0                  155
4   76          True           True  20.14          4.8                  155
5   20         False          False  27.32          6.6                   85

   diabetes  gender\_Female  gender\_Male  smoking\_history\_no\_info
1     False           True        False                     True  \textbackslash{}
2     False          False         True                    False
3     False           True        False                    False
4     False          False         True                    False
5     False           True        False                    False

   smoking\_history\_current  smoking\_history\_ever  smoking\_history\_former
1                    False                 False                   False  \textbackslash{}
2                    False                 False                   False
3                     True                 False                   False
4                     True                 False                   False
5                    False                 False                   False

   smoking\_history\_never  smoking\_history\_not\_current
1                  False                        False
2                   True                        False
3                  False                        False
4                  False                        False
5                   True                        False
\end{Verbatim}
\end{tcolorbox}
        
    Then, we separate our target variable.

    \begin{tcolorbox}[breakable, size=fbox, boxrule=1pt, pad at break*=1mm,colback=cellbackground, colframe=cellborder]
\prompt{In}{incolor}{245}{\boxspacing}
\begin{Verbatim}[commandchars=\\\{\}]
\PY{n}{X} \PY{o}{=} \PY{n}{df\PYZus{}encoded}\PY{o}{.}\PY{n}{drop}\PY{p}{(}\PY{l+s+s1}{\PYZsq{}}\PY{l+s+s1}{diabetes}\PY{l+s+s1}{\PYZsq{}}\PY{p}{,} \PY{n}{axis}\PY{o}{=}\PY{l+m+mi}{1}\PY{p}{)}
\PY{n}{y} \PY{o}{=} \PY{n}{df\PYZus{}encoded}\PY{p}{[}\PY{l+s+s1}{\PYZsq{}}\PY{l+s+s1}{diabetes}\PY{l+s+s1}{\PYZsq{}}\PY{p}{]}
\end{Verbatim}
\end{tcolorbox}

    And split the data into training and testing datasets, with the ratio of
\(80\%\) and \(20\%\). It is good practice to seed the pseudorandom
generator, so it will always result in the same split, across multiple
runs of our notebook.

    \begin{tcolorbox}[breakable, size=fbox, boxrule=1pt, pad at break*=1mm,colback=cellbackground, colframe=cellborder]
\prompt{In}{incolor}{246}{\boxspacing}
\begin{Verbatim}[commandchars=\\\{\}]
\PY{n}{X\PYZus{}train}\PY{p}{,} \PY{n}{X\PYZus{}test}\PY{p}{,} \PY{n}{y\PYZus{}train}\PY{p}{,} \PY{n}{y\PYZus{}test} \PY{o}{=} \PY{n}{train\PYZus{}test\PYZus{}split}\PY{p}{(}\PY{n}{X}\PY{p}{,} \PY{n}{y}\PY{p}{,} \PY{n}{test\PYZus{}size}\PY{o}{=}\PY{l+m+mf}{0.2}\PY{p}{,} \PY{n}{random\PYZus{}state}\PY{o}{=}\PY{l+m+mi}{42}\PY{p}{)}
\end{Verbatim}
\end{tcolorbox}

    Then we create our multivariate linear regression model, using
\texttt{scikit-learn}.

    \begin{tcolorbox}[breakable, size=fbox, boxrule=1pt, pad at break*=1mm,colback=cellbackground, colframe=cellborder]
\prompt{In}{incolor}{247}{\boxspacing}
\begin{Verbatim}[commandchars=\\\{\}]
\PY{n}{model} \PY{o}{=} \PY{n}{LinearRegression}\PY{p}{(}\PY{p}{)}
\PY{n}{\PYZus{}} \PY{o}{=} \PY{n}{model}\PY{o}{.}\PY{n}{fit}\PY{p}{(}\PY{n}{X\PYZus{}train}\PY{p}{,} \PY{n}{y\PYZus{}train}\PY{p}{)}
\end{Verbatim}
\end{tcolorbox}

    The model's coefficients are as follows:

    \begin{tcolorbox}[breakable, size=fbox, boxrule=1pt, pad at break*=1mm,colback=cellbackground, colframe=cellborder]
\prompt{In}{incolor}{248}{\boxspacing}
\begin{Verbatim}[commandchars=\\\{\}]
\PY{n}{coefficients} \PY{o}{=} \PY{n}{pd}\PY{o}{.}\PY{n}{DataFrame}\PY{p}{(}\PY{p}{\PYZob{}}\PY{l+s+s1}{\PYZsq{}}\PY{l+s+s1}{Variable}\PY{l+s+s1}{\PYZsq{}}\PY{p}{:} \PY{n}{X}\PY{o}{.}\PY{n}{columns}\PY{p}{,} \PY{l+s+s1}{\PYZsq{}}\PY{l+s+s1}{Coefficient}\PY{l+s+s1}{\PYZsq{}}\PY{p}{:} \PY{n}{model}\PY{o}{.}\PY{n}{coef\PYZus{}}\PY{p}{\PYZcb{}}\PY{p}{)}\PY{o}{.}\PY{n}{sort\PYZus{}values}\PY{p}{(}\PY{n}{by}\PY{o}{=}\PY{l+s+s1}{\PYZsq{}}\PY{l+s+s1}{Coefficient}\PY{l+s+s1}{\PYZsq{}}\PY{p}{,} \PY{n}{ascending}\PY{o}{=}\PY{k+kc}{False}\PY{p}{)}
\PY{n}{intercept} \PY{o}{=} \PY{n}{pd}\PY{o}{.}\PY{n}{DataFrame}\PY{p}{(}\PY{p}{\PYZob{}}\PY{l+s+s1}{\PYZsq{}}\PY{l+s+s1}{Variable}\PY{l+s+s1}{\PYZsq{}}\PY{p}{:} \PY{p}{[}\PY{l+s+s1}{\PYZsq{}}\PY{l+s+s1}{Intercept}\PY{l+s+s1}{\PYZsq{}}\PY{p}{]}\PY{p}{,} \PY{l+s+s1}{\PYZsq{}}\PY{l+s+s1}{Coefficient}\PY{l+s+s1}{\PYZsq{}}\PY{p}{:} \PY{n}{model}\PY{o}{.}\PY{n}{intercept\PYZus{}}\PY{p}{\PYZcb{}}\PY{p}{)}

\PY{n+nb}{print}\PY{p}{(}\PY{l+s+s2}{\PYZdq{}}\PY{l+s+s2}{Coefficients:}\PY{l+s+s2}{\PYZdq{}}\PY{p}{)}
\PY{n+nb}{print}\PY{p}{(}\PY{n}{coefficients}\PY{p}{)}
\PY{n+nb}{print}\PY{p}{(}\PY{l+s+s2}{\PYZdq{}}\PY{l+s+se}{\PYZbs{}n}\PY{l+s+s2}{Intercept:}\PY{l+s+s2}{\PYZdq{}}\PY{p}{)}
\PY{n+nb}{print}\PY{p}{(}\PY{n}{intercept}\PY{p}{)}
\end{Verbatim}
\end{tcolorbox}

    \begin{Verbatim}[commandchars=\\\{\}]
Coefficients:
                       Variable  Coefficient
2                 heart\_disease     0.133472
1                  hypertension     0.103687
4                   HbA1c\_level     0.076937
11       smoking\_history\_former     0.015804
7                   gender\_Male     0.006450
3                           bmi     0.004050
5           blood\_glucose\_level     0.002163
13  smoking\_history\_not\_current     0.001371
0                           age     0.001358
10         smoking\_history\_ever    -0.001578
9       smoking\_history\_current    -0.001865
12        smoking\_history\_never    -0.003399
6                 gender\_Female    -0.006450
8       smoking\_history\_no\_info    -0.010332

Intercept:
    Variable  Coefficient
0  Intercept    -0.814089
    \end{Verbatim}

    It is important to note here immediately, that the scale of these
coefficients depends on their value sets. For example, gender\_Male is
between \(0\) and \(1\), while BMI is between \(10\) and \(100\). For
this reason, we cannot compare the relative values of the coefficients,
without taking into account the possible values of the variables behind
them.

However, the signs (positive or negative) can be looked at and compared,
which we will do shortly.

    But first, the precision of this model on the test dataset is as
follows:

    \begin{tcolorbox}[breakable, size=fbox, boxrule=1pt, pad at break*=1mm,colback=cellbackground, colframe=cellborder]
\prompt{In}{incolor}{249}{\boxspacing}
\begin{Verbatim}[commandchars=\\\{\}]
\PY{n}{y\PYZus{}pred} \PY{o}{=} \PY{n}{model}\PY{o}{.}\PY{n}{predict}\PY{p}{(}\PY{n}{X\PYZus{}test}\PY{p}{)}
\PY{n}{mse} \PY{o}{=} \PY{n}{mean\PYZus{}squared\PYZus{}error}\PY{p}{(}\PY{n}{y\PYZus{}test}\PY{p}{,} \PY{n}{y\PYZus{}pred}\PY{p}{)}
\PY{n+nb}{print}\PY{p}{(}\PY{l+s+s2}{\PYZdq{}}\PY{l+s+s2}{Mean Squared Error:}\PY{l+s+s2}{\PYZdq{}}\PY{p}{,} \PY{n}{mse}\PY{p}{)}
\end{Verbatim}
\end{tcolorbox}

    \begin{Verbatim}[commandchars=\\\{\}]
Mean Squared Error: 0.04693052558568325
    \end{Verbatim}

    And on the train dataset:

    \begin{tcolorbox}[breakable, size=fbox, boxrule=1pt, pad at break*=1mm,colback=cellbackground, colframe=cellborder]
\prompt{In}{incolor}{250}{\boxspacing}
\begin{Verbatim}[commandchars=\\\{\}]
\PY{n}{y\PYZus{}pred} \PY{o}{=} \PY{n}{model}\PY{o}{.}\PY{n}{predict}\PY{p}{(}\PY{n}{X\PYZus{}train}\PY{p}{)}
\PY{n}{mse} \PY{o}{=} \PY{n}{mean\PYZus{}squared\PYZus{}error}\PY{p}{(}\PY{n}{y\PYZus{}train}\PY{p}{,} \PY{n}{y\PYZus{}pred}\PY{p}{)}
\PY{n+nb}{print}\PY{p}{(}\PY{l+s+s2}{\PYZdq{}}\PY{l+s+s2}{Mean Squared Error:}\PY{l+s+s2}{\PYZdq{}}\PY{p}{,} \PY{n}{mse}\PY{p}{)}
\end{Verbatim}
\end{tcolorbox}

    \begin{Verbatim}[commandchars=\\\{\}]
Mean Squared Error: 0.048474640595590854
    \end{Verbatim}

    We can conclude, that the model was not overfit and the error is
relatively low.

    \hypertarget{independence-testing}{%
\subsection{Independence testing}\label{independence-testing}}

\hypertarget{gender}{%
\subsubsection{Gender}\label{gender}}

Examining the coefficients, the first thing I noticed is the
\texttt{gender} variables. It seems that, with all other variables
present, being male slightly increases, while being female slightly
reduces the risk of diabetes.

I wonder if without the other variables, what could we say about the
influence of gender on diabetes? I will test the independence of the
\texttt{gender} and \texttt{diabetes} variables in the original dataset
using Chi-square test of independence.

At first, I will create the frequency table of their values:

    \begin{tcolorbox}[breakable, size=fbox, boxrule=1pt, pad at break*=1mm,colback=cellbackground, colframe=cellborder]
\prompt{In}{incolor}{251}{\boxspacing}
\begin{Verbatim}[commandchars=\\\{\}]
\PY{n}{crosstab} \PY{o}{=} \PY{n}{pd}\PY{o}{.}\PY{n}{crosstab}\PY{p}{(}\PY{n}{df}\PY{p}{[}\PY{l+s+s1}{\PYZsq{}}\PY{l+s+s1}{gender}\PY{l+s+s1}{\PYZsq{}}\PY{p}{]}\PY{p}{,} \PY{n}{df}\PY{p}{[}\PY{l+s+s1}{\PYZsq{}}\PY{l+s+s1}{diabetes}\PY{l+s+s1}{\PYZsq{}}\PY{p}{]}\PY{p}{)}
\PY{n}{crosstab}
\end{Verbatim}
\end{tcolorbox}

            \begin{tcolorbox}[breakable, size=fbox, boxrule=.5pt, pad at break*=1mm, opacityfill=0]
\prompt{Out}{outcolor}{251}{\boxspacing}
\begin{Verbatim}[commandchars=\\\{\}]
diabetes  False  True
gender
Female    51188   3856
Male      35697   3620
\end{Verbatim}
\end{tcolorbox}
        
    Then, I will use \texttt{scipy}, to compute a Pearson's chi-squared
statistic.

The null hypothesis is that the variables are independent.

    \begin{tcolorbox}[breakable, size=fbox, boxrule=1pt, pad at break*=1mm,colback=cellbackground, colframe=cellborder]
\prompt{In}{incolor}{252}{\boxspacing}
\begin{Verbatim}[commandchars=\\\{\}]
\PY{n}{chi2}\PY{p}{,} \PY{n}{p}\PY{p}{,} \PY{n}{dof}\PY{p}{,} \PY{n}{expected} \PY{o}{=} \PY{n}{chi2\PYZus{}contingency}\PY{p}{(}\PY{n}{crosstab}\PY{p}{)}
\PY{n+nb}{print}\PY{p}{(}\PY{l+s+sa}{f}\PY{l+s+s2}{\PYZdq{}}\PY{l+s+s2}{Chi\PYZhy{}square value: }\PY{l+s+si}{\PYZob{}}\PY{n}{chi2}\PY{l+s+si}{\PYZcb{}}\PY{l+s+s2}{\PYZdq{}}\PY{p}{)}
\PY{n+nb}{print}\PY{p}{(}\PY{l+s+sa}{f}\PY{l+s+s2}{\PYZdq{}}\PY{l+s+s2}{P\PYZhy{}value: }\PY{l+s+si}{\PYZob{}}\PY{n}{p}\PY{l+s+si}{\PYZcb{}}\PY{l+s+s2}{\PYZdq{}}\PY{p}{)}
\PY{n+nb}{print}\PY{p}{(}\PY{l+s+sa}{f}\PY{l+s+s2}{\PYZdq{}}\PY{l+s+s2}{Degrees of freedom: }\PY{l+s+si}{\PYZob{}}\PY{n}{dof}\PY{l+s+si}{\PYZcb{}}\PY{l+s+s2}{\PYZdq{}}\PY{p}{)}
\PY{n+nb}{print}\PY{p}{(}\PY{l+s+s2}{\PYZdq{}}\PY{l+s+s2}{Expected:}\PY{l+s+s2}{\PYZdq{}}\PY{p}{)}
\PY{n+nb}{print}\PY{p}{(}\PY{n}{expected}\PY{p}{)}
\end{Verbatim}
\end{tcolorbox}

    \begin{Verbatim}[commandchars=\\\{\}]
Chi-square value: 152.1271942474673
P-value: 5.94328803769177e-35
Degrees of freedom: 1
Expected:
[[50682.9933977  4361.0066023]
 [36202.0066023  3114.9933977]]
    \end{Verbatim}

    With a p-value of \(p=5,94\cdot{}10^{-35}\) we can say, that we have
very strong evidence against our null hypothesis, therefore we can
conclude that gender and diabetes are indeed dependent:

The gender of a person influences the likelihood of developing diabetes.

As we can see in the Expected matrix, that is the expected frequencies
for the same population size, were these variables independent of each
other. We can see that we got less than expected females with diabetes
and more than expected males with diabetes.

    \hypertarget{smoking-history}{%
\subsubsection{Smoking history}\label{smoking-history}}

The other interesting thing I noticed about the multivariate linear
regression coefficients is that \texttt{smoking\_history\_current} has a
negative coefficient, while \texttt{smoking\_history\_former} and
\texttt{smoking\_history\_not\_current} have positive coefficients. At
least \texttt{smoking\_history\_never} has a negative coefficient too,
which makes perfect sense.

I do have tales from friends and family saying that when someone chooses
to stop smoking is when the problems start. A more likely explanation is
that when people stop smoking, the damage has already been done. When
someone is still smoking, maybe they have less years behind them and
quitting smoking \textbf{earlier} should help lower the risk of diabetes
and heart disease (amongst many diseases).

I will also run the same independence test on \texttt{smoking\_history}
and \texttt{diabetes}:

    \begin{tcolorbox}[breakable, size=fbox, boxrule=1pt, pad at break*=1mm,colback=cellbackground, colframe=cellborder]
\prompt{In}{incolor}{253}{\boxspacing}
\begin{Verbatim}[commandchars=\\\{\}]
\PY{n}{crosstab} \PY{o}{=} \PY{n}{pd}\PY{o}{.}\PY{n}{crosstab}\PY{p}{(}\PY{n}{df}\PY{p}{[}\PY{l+s+s1}{\PYZsq{}}\PY{l+s+s1}{smoking\PYZus{}history}\PY{l+s+s1}{\PYZsq{}}\PY{p}{]}\PY{p}{,} \PY{n}{df}\PY{p}{[}\PY{l+s+s1}{\PYZsq{}}\PY{l+s+s1}{diabetes}\PY{l+s+s1}{\PYZsq{}}\PY{p}{]}\PY{p}{)}
\PY{n}{crosstab}
\end{Verbatim}
\end{tcolorbox}

            \begin{tcolorbox}[breakable, size=fbox, boxrule=.5pt, pad at break*=1mm, opacityfill=0]
\prompt{Out}{outcolor}{253}{\boxspacing}
\begin{Verbatim}[commandchars=\\\{\}]
diabetes         False  True
smoking\_history
no\_info          32751   1260
current           8227    924
ever              3376    436
former            7004   1372
never            30181   2878
not\_current       5346    606
\end{Verbatim}
\end{tcolorbox}
        
    \begin{tcolorbox}[breakable, size=fbox, boxrule=1pt, pad at break*=1mm,colback=cellbackground, colframe=cellborder]
\prompt{In}{incolor}{254}{\boxspacing}
\begin{Verbatim}[commandchars=\\\{\}]
\PY{n}{chi2}\PY{p}{,} \PY{n}{p}\PY{p}{,} \PY{n}{dof}\PY{p}{,} \PY{n}{expected} \PY{o}{=} \PY{n}{chi2\PYZus{}contingency}\PY{p}{(}\PY{n}{crosstab}\PY{p}{)}
\PY{n+nb}{print}\PY{p}{(}\PY{l+s+sa}{f}\PY{l+s+s2}{\PYZdq{}}\PY{l+s+s2}{Chi\PYZhy{}square value: }\PY{l+s+si}{\PYZob{}}\PY{n}{chi2}\PY{l+s+si}{\PYZcb{}}\PY{l+s+s2}{\PYZdq{}}\PY{p}{)}
\PY{n+nb}{print}\PY{p}{(}\PY{l+s+sa}{f}\PY{l+s+s2}{\PYZdq{}}\PY{l+s+s2}{P\PYZhy{}value: }\PY{l+s+si}{\PYZob{}}\PY{n}{p}\PY{l+s+si}{\PYZcb{}}\PY{l+s+s2}{\PYZdq{}}\PY{p}{)}
\PY{n+nb}{print}\PY{p}{(}\PY{l+s+sa}{f}\PY{l+s+s2}{\PYZdq{}}\PY{l+s+s2}{Degrees of freedom: }\PY{l+s+si}{\PYZob{}}\PY{n}{dof}\PY{l+s+si}{\PYZcb{}}\PY{l+s+s2}{\PYZdq{}}\PY{p}{)}
\PY{n+nb}{print}\PY{p}{(}\PY{l+s+s2}{\PYZdq{}}\PY{l+s+s2}{Expected:}\PY{l+s+s2}{\PYZdq{}}\PY{p}{)}
\PY{n+nb}{print}\PY{p}{(}\PY{n}{expected}\PY{p}{)}
\end{Verbatim}
\end{tcolorbox}

    \begin{Verbatim}[commandchars=\\\{\}]
Chi-square value: 1844.0298560488386
P-value: 0.0
Degrees of freedom: 5
Expected:
[[31316.38849737  2694.61150263]
 [ 8425.98780216   725.01219784]
 [ 3509.98420958   302.01579042]
 [ 7712.3892286    663.6107714 ]
 [30439.81321732  2619.18678268]
 [ 5480.43704497   471.56295503]]
    \end{Verbatim}

    The p-value is so small, the numerical representation displays it as
\(0\). This means smoking history has indeed an influence on whether or
not someone develops diabetes.

I would like to further investigate this relationship. Since diabetes is
a binary variable, it is hard to see exactly the effects of smoking on
it. However, we have a few numerical variables in the dataset, which are
strong indicators of diabetes, such as the various measurements of blood
levels and BMI.

So I will continue exploring the effects of smoking on these variables
further.

    \hypertarget{correlation-matrix}{%
\subsection{Correlation matrix}\label{correlation-matrix}}

    In order to establish which numerical variables have a strong
relationship with diabetes, I will be using a correlation matrix heatmap
to visualise these 1-1 relationships.

    \begin{tcolorbox}[breakable, size=fbox, boxrule=1pt, pad at break*=1mm,colback=cellbackground, colframe=cellborder]
\prompt{In}{incolor}{255}{\boxspacing}
\begin{Verbatim}[commandchars=\\\{\}]
\PY{n}{correlation\PYZus{}matrix} \PY{o}{=} \PY{n}{df\PYZus{}encoded}\PY{o}{.}\PY{n}{corr}\PY{p}{(}\PY{p}{)}

\PY{n}{plt}\PY{o}{.}\PY{n}{figure}\PY{p}{(}\PY{n}{figsize}\PY{o}{=}\PY{p}{(}\PY{l+m+mi}{15}\PY{p}{,} \PY{l+m+mi}{10}\PY{p}{)}\PY{p}{)}
\PY{n}{sns}\PY{o}{.}\PY{n}{heatmap}\PY{p}{(}\PY{n}{correlation\PYZus{}matrix}\PY{p}{,} \PY{n}{annot}\PY{o}{=}\PY{k+kc}{True}\PY{p}{,} \PY{n}{cmap}\PY{o}{=}\PY{l+s+s1}{\PYZsq{}}\PY{l+s+s1}{coolwarm}\PY{l+s+s1}{\PYZsq{}}\PY{p}{,} \PY{n}{center}\PY{o}{=}\PY{l+m+mi}{0}\PY{p}{)}

\PY{n}{plt}\PY{o}{.}\PY{n}{title}\PY{p}{(}\PY{l+s+s1}{\PYZsq{}}\PY{l+s+s1}{Correlation Matrix}\PY{l+s+s1}{\PYZsq{}}\PY{p}{)}
\PY{n}{plt}\PY{o}{.}\PY{n}{show}\PY{p}{(}\PY{p}{)}
\end{Verbatim}
\end{tcolorbox}

    \begin{center}
    \adjustimage{max size={0.9\linewidth}{0.9\paperheight}}{diabetes_files/diabetes_88_0.png}
    \end{center}
    { \hspace*{\fill} \\}
    
    We can see on the heatmap, that \texttt{blood\_glucose\_level} and
\texttt{HbA1c\_level} have the strongest correlation to diabetes, so I
will be exploring the effects of smoking on these variables.

    \hypertarget{testing-the-mean-of-categories}{%
\subsection{Testing the mean of
categories}\label{testing-the-mean-of-categories}}

Smoking history is a categorical variable, which means it defines
different groups of people. For these groups, I can calculate the mean
of \texttt{blood\_glucose\_level} and \texttt{HbA1c\_level}, then, I can
investigate whether or not the differences in the results are
statistically significant.

    \begin{tcolorbox}[breakable, size=fbox, boxrule=1pt, pad at break*=1mm,colback=cellbackground, colframe=cellborder]
\prompt{In}{incolor}{256}{\boxspacing}
\begin{Verbatim}[commandchars=\\\{\}]
\PY{n}{df}\PY{o}{.}\PY{n}{groupby}\PY{p}{(}\PY{l+s+s1}{\PYZsq{}}\PY{l+s+s1}{smoking\PYZus{}history}\PY{l+s+s1}{\PYZsq{}}\PY{p}{)}\PY{p}{[}\PY{p}{[}\PY{l+s+s1}{\PYZsq{}}\PY{l+s+s1}{blood\PYZus{}glucose\PYZus{}level}\PY{l+s+s1}{\PYZsq{}}\PY{p}{]}\PY{p}{]}\PY{o}{.}\PY{n}{mean}\PY{p}{(}\PY{p}{)}\PY{o}{.}\PY{n}{sort\PYZus{}values}\PY{p}{(}\PY{l+s+s1}{\PYZsq{}}\PY{l+s+s1}{blood\PYZus{}glucose\PYZus{}level}\PY{l+s+s1}{\PYZsq{}}\PY{p}{)}
\end{Verbatim}
\end{tcolorbox}

            \begin{tcolorbox}[breakable, size=fbox, boxrule=.5pt, pad at break*=1mm, opacityfill=0]
\prompt{Out}{outcolor}{256}{\boxspacing}
\begin{Verbatim}[commandchars=\\\{\}]
                 blood\_glucose\_level
smoking\_history
no\_info                   135.019729
never                     138.245107
ever                      139.076600
not\_current               139.088878
current                   139.614905
former                    142.896968
\end{Verbatim}
\end{tcolorbox}
        
    Here, we can see that the average blood glucose level slightly
increases, as someone has more history with smoking. Interestingly
again, current smokers rank below former smokers. This can again be due
to the fact that former smokers may have a longer smoking history than
current smokers.

    \begin{tcolorbox}[breakable, size=fbox, boxrule=1pt, pad at break*=1mm,colback=cellbackground, colframe=cellborder]
\prompt{In}{incolor}{257}{\boxspacing}
\begin{Verbatim}[commandchars=\\\{\}]
\PY{n}{df}\PY{o}{.}\PY{n}{groupby}\PY{p}{(}\PY{l+s+s1}{\PYZsq{}}\PY{l+s+s1}{smoking\PYZus{}history}\PY{l+s+s1}{\PYZsq{}}\PY{p}{)}\PY{p}{[}\PY{p}{[}\PY{l+s+s1}{\PYZsq{}}\PY{l+s+s1}{HbA1c\PYZus{}level}\PY{l+s+s1}{\PYZsq{}}\PY{p}{]}\PY{p}{]}\PY{o}{.}\PY{n}{mean}\PY{p}{(}\PY{p}{)}\PY{o}{.}\PY{n}{sort\PYZus{}values}\PY{p}{(}\PY{l+s+s1}{\PYZsq{}}\PY{l+s+s1}{HbA1c\PYZus{}level}\PY{l+s+s1}{\PYZsq{}}\PY{p}{)}
\end{Verbatim}
\end{tcolorbox}

            \begin{tcolorbox}[breakable, size=fbox, boxrule=.5pt, pad at break*=1mm, opacityfill=0]
\prompt{Out}{outcolor}{257}{\boxspacing}
\begin{Verbatim}[commandchars=\\\{\}]
                 HbA1c\_level
smoking\_history
no\_info             5.457690
never               5.531640
current             5.546694
not\_current         5.557023
ever                5.571800
former              5.646036
\end{Verbatim}
\end{tcolorbox}
        
    However, for \texttt{HbA1c\_level}, the level of \texttt{ever}, which I
assume means ``smoked a bit a long time ago'' seems really high on the
ranking.

I would like to know if there is a statistically significant difference
between the average \texttt{HbA1c\_level}s of the
\texttt{smoking\_history} categories. ANOVA would not tell me which
specific groups are different, so instead I will perform pairwise
hypothesis testing for the equivalence of the mean between all
\(5\cdot{}5\) category pairs and display the resulting p-values in a
matrix heatmap format, similar to the correlation matrix.

The diagonal of this matrix will be comparing the same variables, but
this is also the case for the correlation matrix, so I will keep those
values in as well.

I could use paired t-tests, but that would require knowing that the
variances are the same. I will be using Welch-test instead, which is
more general and has a similar built-in function in scipy.

I will be calling this the Welch p-value matrix. I have tried finding
such a thing in the literature, but I wasn't able to. For example MANOVA
tests multiple dependent variables, while I have a single dependent
variable and I want all pairwise relations between the various
categories.

Let us first define the function:

    \begin{tcolorbox}[breakable, size=fbox, boxrule=1pt, pad at break*=1mm,colback=cellbackground, colframe=cellborder]
\prompt{In}{incolor}{258}{\boxspacing}
\begin{Verbatim}[commandchars=\\\{\}]
\PY{k}{def} \PY{n+nf}{welch\PYZus{}p\PYZus{}value\PYZus{}matrix}\PY{p}{(}\PY{n}{category}\PY{p}{,} \PY{n}{target}\PY{p}{)}\PY{p}{:}
  \PY{n}{categories} \PY{o}{=} \PY{n}{df}\PY{p}{[}\PY{n}{category}\PY{p}{]}\PY{o}{.}\PY{n}{unique}\PY{p}{(}\PY{p}{)}

  \PY{n}{p\PYZus{}val\PYZus{}matrix} \PY{o}{=} \PY{n}{np}\PY{o}{.}\PY{n}{zeros}\PY{p}{(}\PY{p}{(}\PY{n+nb}{len}\PY{p}{(}\PY{n}{categories}\PY{p}{)}\PY{p}{,} \PY{n+nb}{len}\PY{p}{(}\PY{n}{categories}\PY{p}{)}\PY{p}{)}\PY{p}{)}

  \PY{k}{for} \PY{n}{i}\PY{p}{,} \PY{n}{category\PYZus{}i} \PY{o+ow}{in} \PY{n+nb}{enumerate}\PY{p}{(}\PY{n}{categories}\PY{p}{)}\PY{p}{:}
    \PY{k}{for} \PY{n}{j}\PY{p}{,} \PY{n}{category\PYZus{}j} \PY{o+ow}{in} \PY{n+nb}{enumerate}\PY{p}{(}\PY{n}{categories}\PY{p}{)}\PY{p}{:}
    
      \PY{n}{category\PYZus{}data\PYZus{}i} \PY{o}{=} \PY{n}{df}\PY{p}{[}\PY{n}{df}\PY{p}{[}\PY{n}{category}\PY{p}{]} \PY{o}{==} \PY{n}{category\PYZus{}i}\PY{p}{]}\PY{p}{[}\PY{n}{target}\PY{p}{]}
      \PY{n}{category\PYZus{}data\PYZus{}j} \PY{o}{=} \PY{n}{df}\PY{p}{[}\PY{n}{df}\PY{p}{[}\PY{n}{category}\PY{p}{]} \PY{o}{==} \PY{n}{category\PYZus{}j}\PY{p}{]}\PY{p}{[}\PY{n}{target}\PY{p}{]}
    
      \PY{c+c1}{\PYZsh{} Performs Welch\PYZsq{}s t\PYZhy{}test, when equal\PYZus{}var = False is set.}
      \PY{n}{t\PYZus{}stat}\PY{p}{,} \PY{n}{p\PYZus{}val} \PY{o}{=} \PY{n}{ttest\PYZus{}ind}\PY{p}{(}\PY{n}{category\PYZus{}data\PYZus{}i}\PY{p}{,} \PY{n}{category\PYZus{}data\PYZus{}j}\PY{p}{,} \PY{n}{equal\PYZus{}var}\PY{o}{=}\PY{k+kc}{False}\PY{p}{)}
    
      \PY{n}{p\PYZus{}val\PYZus{}matrix}\PY{p}{[}\PY{n}{i}\PY{p}{,} \PY{n}{j}\PY{p}{]} \PY{o}{=} \PY{n}{p\PYZus{}val}

  \PY{n}{plt}\PY{o}{.}\PY{n}{figure}\PY{p}{(}\PY{n}{figsize}\PY{o}{=}\PY{p}{(}\PY{l+m+mi}{10}\PY{p}{,} \PY{l+m+mi}{4}\PY{p}{)}\PY{p}{)}

  \PY{n}{sns}\PY{o}{.}\PY{n}{heatmap}\PY{p}{(}\PY{n}{p\PYZus{}val\PYZus{}matrix}\PY{p}{,} \PY{n}{cmap}\PY{o}{=}\PY{l+s+s1}{\PYZsq{}}\PY{l+s+s1}{YlOrRd}\PY{l+s+s1}{\PYZsq{}}\PY{p}{,} \PY{n}{annot}\PY{o}{=}\PY{k+kc}{True}\PY{p}{,} \PY{n}{fmt}\PY{o}{=}\PY{l+s+s2}{\PYZdq{}}\PY{l+s+s2}{.2g}\PY{l+s+s2}{\PYZdq{}}\PY{p}{,} 
            \PY{n}{xticklabels}\PY{o}{=}\PY{n}{categories}\PY{p}{,} \PY{n}{yticklabels}\PY{o}{=}\PY{n}{categories}\PY{p}{,} \PY{n}{vmax}\PY{o}{=}\PY{l+m+mf}{0.1}\PY{p}{,}
            \PY{n}{cbar\PYZus{}kws}\PY{o}{=}\PY{p}{\PYZob{}}\PY{l+s+s1}{\PYZsq{}}\PY{l+s+s1}{label}\PY{l+s+s1}{\PYZsq{}}\PY{p}{:} \PY{l+s+s1}{\PYZsq{}}\PY{l+s+s1}{p\PYZhy{}value}\PY{l+s+s1}{\PYZsq{}}\PY{p}{\PYZcb{}}\PY{p}{)}

  \PY{n}{plt}\PY{o}{.}\PY{n}{title}\PY{p}{(}\PY{l+s+sa}{f}\PY{l+s+s2}{\PYZdq{}}\PY{l+s+s2}{Welch p\PYZhy{}value matrix for categories }\PY{l+s+si}{\PYZob{}}\PY{n}{category}\PY{l+s+si}{\PYZcb{}}\PY{l+s+s2}{ and dependent variable }\PY{l+s+si}{\PYZob{}}\PY{n}{target}\PY{l+s+si}{\PYZcb{}}\PY{l+s+s2}{\PYZdq{}}\PY{p}{)}
  \PY{n}{plt}\PY{o}{.}\PY{n}{show}\PY{p}{(}\PY{p}{)}
\end{Verbatim}
\end{tcolorbox}

    Then let's run this function for the \texttt{HbA1c\_level}!

    \begin{tcolorbox}[breakable, size=fbox, boxrule=1pt, pad at break*=1mm,colback=cellbackground, colframe=cellborder]
\prompt{In}{incolor}{259}{\boxspacing}
\begin{Verbatim}[commandchars=\\\{\}]
\PY{n}{welch\PYZus{}p\PYZus{}value\PYZus{}matrix}\PY{p}{(}\PY{l+s+s1}{\PYZsq{}}\PY{l+s+s1}{smoking\PYZus{}history}\PY{l+s+s1}{\PYZsq{}}\PY{p}{,} \PY{l+s+s1}{\PYZsq{}}\PY{l+s+s1}{HbA1c\PYZus{}level}\PY{l+s+s1}{\PYZsq{}}\PY{p}{)}
\end{Verbatim}
\end{tcolorbox}

    \begin{center}
    \adjustimage{max size={0.9\linewidth}{0.9\paperheight}}{diabetes_files/diabetes_97_0.png}
    \end{center}
    { \hspace*{\fill} \\}
    
    If the cell is red, it means there is no statistically significant
difference between the two categories for the \texttt{HbA1c\_level}. If
the cell is orange, we can say that there is a statistically significant
difference (\(p<0.05\)). Finally, when the cell is light yellow, the
\(p\) value is even smaller, \(p<0.01\).

Interestingly, if someone is a \texttt{former} smoker, which I assume
means ``has been addicted to smoking and quit'', the increase in the
average \texttt{HbA1c\_level}s is statistically significant, relative to
all other groups, including current smokers!

Other than this, \texttt{no\_info} had a similar result. Since
\texttt{no\_info} has the smallest average among all of the categories,
so I believed \texttt{no\_info} should be similar to \texttt{never},
however they statistially differ.

Finally, \texttt{ever} and \texttt{never} also have a statistically
significant difference, \texttt{ever} having the higher average between
the two. So if someone has smoked in the past, that can still put them
in a higher risk category for increased \texttt{HbA1c\_level}s.

    Now, let us look at the target variable \texttt{blood\_glucose\_level}!

    \begin{tcolorbox}[breakable, size=fbox, boxrule=1pt, pad at break*=1mm,colback=cellbackground, colframe=cellborder]
\prompt{In}{incolor}{260}{\boxspacing}
\begin{Verbatim}[commandchars=\\\{\}]
\PY{n}{welch\PYZus{}p\PYZus{}value\PYZus{}matrix}\PY{p}{(}\PY{l+s+s1}{\PYZsq{}}\PY{l+s+s1}{smoking\PYZus{}history}\PY{l+s+s1}{\PYZsq{}}\PY{p}{,} \PY{l+s+s1}{\PYZsq{}}\PY{l+s+s1}{blood\PYZus{}glucose\PYZus{}level}\PY{l+s+s1}{\PYZsq{}}\PY{p}{)}
\end{Verbatim}
\end{tcolorbox}

    \begin{center}
    \adjustimage{max size={0.9\linewidth}{0.9\paperheight}}{diabetes_files/diabetes_100_0.png}
    \end{center}
    { \hspace*{\fill} \\}
    
    We have similar results for the categories \texttt{no\_info} and
\texttt{former}. However, this time \texttt{current} and \texttt{never}
also have a statistically significant difference, \texttt{current}
having an increased average \texttt{blood\_glucose\_level}.

    After looking at these results, my conclusion is that categorical
representation of \texttt{smoking\_history} seems unfit for statistical
purposes. A better way to represent smoking would be giving the number
of months smoked during someone's lifetime and the number of months
after their last smoke. I believe these variables would better represent
the differences than these categories.

    \hypertarget{selected-multivariate-linear-regression}{%
\subsection{Selected multivariate linear
regression}\label{selected-multivariate-linear-regression}}

Finally, I would like to form the best multivariate regression model for
predicting diabetes.

How do I define best?

I want to use only a subset of the variables, minimizing the number of
them needed, which still do a relatively good job at minimizing the MSE.

In order to find the best subset, I will first calculate the MSE for all
subset of variables (the power set of the columns), then I will plot
these against the number of variables needed, on a scatter plot.

First, I have created a function, that can do the same linear regression
I did previously, but on the variables, specified on the input
parameter:

    \begin{tcolorbox}[breakable, size=fbox, boxrule=1pt, pad at break*=1mm,colback=cellbackground, colframe=cellborder]
\prompt{In}{incolor}{264}{\boxspacing}
\begin{Verbatim}[commandchars=\\\{\}]
\PY{k}{def} \PY{n+nf}{linear\PYZus{}regression}\PY{p}{(}\PY{n}{variables}\PY{p}{,} \PY{n}{should\PYZus{}print}\PY{o}{=}\PY{k+kc}{False}\PY{p}{)}\PY{p}{:}

  \PY{n}{X} \PY{o}{=} \PY{n}{df\PYZus{}encoded}\PY{p}{[}\PY{n}{variables}\PY{p}{]}
  \PY{n}{y} \PY{o}{=} \PY{n}{df\PYZus{}encoded}\PY{p}{[}\PY{l+s+s1}{\PYZsq{}}\PY{l+s+s1}{diabetes}\PY{l+s+s1}{\PYZsq{}}\PY{p}{]}

  \PY{n}{X\PYZus{}train}\PY{p}{,} \PY{n}{X\PYZus{}test}\PY{p}{,} \PY{n}{y\PYZus{}train}\PY{p}{,} \PY{n}{y\PYZus{}test} \PY{o}{=} \PY{n}{train\PYZus{}test\PYZus{}split}\PY{p}{(}\PY{n}{X}\PY{p}{,} \PY{n}{y}\PY{p}{,} \PY{n}{test\PYZus{}size}\PY{o}{=}\PY{l+m+mf}{0.2}\PY{p}{,} \PY{n}{random\PYZus{}state}\PY{o}{=}\PY{l+m+mi}{42}\PY{p}{)}
  
  \PY{n}{model} \PY{o}{=} \PY{n}{LinearRegression}\PY{p}{(}\PY{p}{)}
  \PY{n}{model}\PY{o}{.}\PY{n}{fit}\PY{p}{(}\PY{n}{X\PYZus{}train}\PY{p}{,} \PY{n}{y\PYZus{}train}\PY{p}{)}
  
  \PY{n}{coefficients} \PY{o}{=} \PY{n}{pd}\PY{o}{.}\PY{n}{DataFrame}\PY{p}{(}\PY{p}{\PYZob{}}\PY{l+s+s1}{\PYZsq{}}\PY{l+s+s1}{Variable}\PY{l+s+s1}{\PYZsq{}}\PY{p}{:} \PY{n}{X}\PY{o}{.}\PY{n}{columns}\PY{p}{,} \PY{l+s+s1}{\PYZsq{}}\PY{l+s+s1}{Coefficient}\PY{l+s+s1}{\PYZsq{}}\PY{p}{:} \PY{n}{model}\PY{o}{.}\PY{n}{coef\PYZus{}}\PY{p}{\PYZcb{}}\PY{p}{)}\PY{o}{.}\PY{n}{sort\PYZus{}values}\PY{p}{(}\PY{n}{by}\PY{o}{=}\PY{l+s+s1}{\PYZsq{}}\PY{l+s+s1}{Coefficient}\PY{l+s+s1}{\PYZsq{}}\PY{p}{,} \PY{n}{ascending}\PY{o}{=}\PY{k+kc}{False}\PY{p}{)}
  \PY{n}{intercept} \PY{o}{=} \PY{n}{pd}\PY{o}{.}\PY{n}{DataFrame}\PY{p}{(}\PY{p}{\PYZob{}}\PY{l+s+s1}{\PYZsq{}}\PY{l+s+s1}{Variable}\PY{l+s+s1}{\PYZsq{}}\PY{p}{:} \PY{p}{[}\PY{l+s+s1}{\PYZsq{}}\PY{l+s+s1}{Intercept}\PY{l+s+s1}{\PYZsq{}}\PY{p}{]}\PY{p}{,} \PY{l+s+s1}{\PYZsq{}}\PY{l+s+s1}{Coefficient}\PY{l+s+s1}{\PYZsq{}}\PY{p}{:} \PY{n}{model}\PY{o}{.}\PY{n}{intercept\PYZus{}}\PY{p}{\PYZcb{}}\PY{p}{)}

  \PY{n}{y\PYZus{}pred} \PY{o}{=} \PY{n}{model}\PY{o}{.}\PY{n}{predict}\PY{p}{(}\PY{n}{X\PYZus{}test}\PY{p}{)}
  \PY{n}{mse} \PY{o}{=} \PY{n}{mean\PYZus{}squared\PYZus{}error}\PY{p}{(}\PY{n}{y\PYZus{}test}\PY{p}{,} \PY{n}{y\PYZus{}pred}\PY{p}{)}

  \PY{k}{if} \PY{n}{should\PYZus{}print}\PY{p}{:}
    \PY{n+nb}{print}\PY{p}{(}\PY{l+s+s2}{\PYZdq{}}\PY{l+s+s2}{Coefficients:}\PY{l+s+s2}{\PYZdq{}}\PY{p}{)}
    \PY{n+nb}{print}\PY{p}{(}\PY{n}{coefficients}\PY{p}{)}
    \PY{n+nb}{print}\PY{p}{(}\PY{p}{)}
    \PY{n+nb}{print}\PY{p}{(}\PY{l+s+s2}{\PYZdq{}}\PY{l+s+s2}{Intercept:}\PY{l+s+s2}{\PYZdq{}}\PY{p}{)}
    \PY{n+nb}{print}\PY{p}{(}\PY{n}{intercept}\PY{p}{)}
    \PY{n+nb}{print}\PY{p}{(}\PY{p}{)}
    \PY{n+nb}{print}\PY{p}{(}\PY{l+s+s2}{\PYZdq{}}\PY{l+s+s2}{Mean Squared Error:}\PY{l+s+s2}{\PYZdq{}}\PY{p}{)}
    \PY{n+nb}{print}\PY{p}{(}\PY{n}{mse}\PY{p}{)}

  \PY{k}{return} \PY{n}{mse}
\end{Verbatim}
\end{tcolorbox}

    Then I ran this for all column subsets (excluding the empty one) and
collected the results:

    \begin{tcolorbox}[breakable, size=fbox, boxrule=1pt, pad at break*=1mm,colback=cellbackground, colframe=cellborder]
\prompt{In}{incolor}{265}{\boxspacing}
\begin{Verbatim}[commandchars=\\\{\}]
\PY{n}{column\PYZus{}names} \PY{o}{=} \PY{n}{df\PYZus{}encoded}\PY{o}{.}\PY{n}{drop}\PY{p}{(}\PY{l+s+s1}{\PYZsq{}}\PY{l+s+s1}{diabetes}\PY{l+s+s1}{\PYZsq{}}\PY{p}{,} \PY{n}{axis}\PY{o}{=}\PY{l+m+mi}{1}\PY{p}{)}\PY{o}{.}\PY{n}{columns}
\PY{n}{power\PYZus{}set\PYZus{}nonempty} \PY{o}{=} \PY{p}{[}\PY{n+nb}{list}\PY{p}{(}\PY{n}{combo}\PY{p}{)} \PY{k}{for} \PY{n}{r} \PY{o+ow}{in} \PY{n+nb}{range}\PY{p}{(}\PY{l+m+mi}{1}\PY{p}{,} \PY{n+nb}{len}\PY{p}{(}\PY{n}{column\PYZus{}names}\PY{p}{)} \PY{o}{+} \PY{l+m+mi}{1}\PY{p}{)} \PY{k}{for} \PY{n}{combo} \PY{o+ow}{in} \PY{n}{combinations}\PY{p}{(}\PY{n}{column\PYZus{}names}\PY{p}{,} \PY{n}{r}\PY{p}{)}\PY{p}{]}

\PY{n}{results} \PY{o}{=} \PY{p}{[}\PY{p}{]}

\PY{k}{for} \PY{n}{columns} \PY{o+ow}{in} \PY{n}{power\PYZus{}set\PYZus{}nonempty}\PY{p}{:}
  \PY{n}{mse} \PY{o}{=} \PY{n}{linear\PYZus{}regression}\PY{p}{(}\PY{n}{columns}\PY{p}{)}
  \PY{n}{results}\PY{o}{.}\PY{n}{append}\PY{p}{(}\PY{p}{\PYZob{}}
    \PY{l+s+s1}{\PYZsq{}}\PY{l+s+s1}{columns}\PY{l+s+s1}{\PYZsq{}}\PY{p}{:} \PY{n}{columns}\PY{p}{,}
    \PY{l+s+s1}{\PYZsq{}}\PY{l+s+s1}{num\PYZus{}vars}\PY{l+s+s1}{\PYZsq{}}\PY{p}{:} \PY{n+nb}{len}\PY{p}{(}\PY{n}{columns}\PY{p}{)}\PY{p}{,}
    \PY{l+s+s1}{\PYZsq{}}\PY{l+s+s1}{mse}\PY{l+s+s1}{\PYZsq{}}\PY{p}{:} \PY{n}{mse}
  \PY{p}{\PYZcb{}}\PY{p}{)}
\end{Verbatim}
\end{tcolorbox}

    Using the following code, I was able to display the actual variables
needed as hover text for every data point. However, this plot is only
visible in the Jupyter notebook it is not displayed when I convert it to
a PDF. For the PDF, a similar plot, without the hover follows next.

    \begin{tcolorbox}[breakable, size=fbox, boxrule=1pt, pad at break*=1mm,colback=cellbackground, colframe=cellborder]
\prompt{In}{incolor}{275}{\boxspacing}
\begin{Verbatim}[commandchars=\\\{\}]
\PY{n}{results\PYZus{}df} \PY{o}{=} \PY{n}{pd}\PY{o}{.}\PY{n}{DataFrame}\PY{p}{(}\PY{n}{results}\PY{p}{)}

\PY{n}{fig} \PY{o}{=} \PY{n}{px}\PY{o}{.}\PY{n}{scatter}\PY{p}{(}\PY{n}{results\PYZus{}df}\PY{p}{,} \PY{n}{x}\PY{o}{=}\PY{l+s+s1}{\PYZsq{}}\PY{l+s+s1}{mse}\PY{l+s+s1}{\PYZsq{}}\PY{p}{,} \PY{n}{y}\PY{o}{=}\PY{l+s+s1}{\PYZsq{}}\PY{l+s+s1}{num\PYZus{}vars}\PY{l+s+s1}{\PYZsq{}}\PY{p}{,} \PY{n}{hover\PYZus{}data}\PY{o}{=}\PY{p}{[}\PY{l+s+s1}{\PYZsq{}}\PY{l+s+s1}{columns}\PY{l+s+s1}{\PYZsq{}}\PY{p}{]}\PY{p}{)}
\PY{n}{fig}\PY{o}{.}\PY{n}{update\PYZus{}layout}\PY{p}{(}\PY{n}{title}\PY{o}{=}\PY{l+s+s1}{\PYZsq{}}\PY{l+s+s1}{MSE relative to the number of variables used}\PY{l+s+s1}{\PYZsq{}}\PY{p}{,} \PY{n}{xaxis\PYZus{}title}\PY{o}{=}\PY{l+s+s1}{\PYZsq{}}\PY{l+s+s1}{MSE}\PY{l+s+s1}{\PYZsq{}}\PY{p}{,} \PY{n}{yaxis\PYZus{}title}\PY{o}{=}\PY{l+s+s1}{\PYZsq{}}\PY{l+s+s1}{Number of variables}\PY{l+s+s1}{\PYZsq{}}\PY{p}{)}
\PY{n}{fig}\PY{o}{.}\PY{n}{show}\PY{p}{(}\PY{p}{)}
\end{Verbatim}
\end{tcolorbox}

    
    
    For the PDF, here is the same plot, without the hover text:

    \begin{tcolorbox}[breakable, size=fbox, boxrule=1pt, pad at break*=1mm,colback=cellbackground, colframe=cellborder]
\prompt{In}{incolor}{276}{\boxspacing}
\begin{Verbatim}[commandchars=\\\{\}]
\PY{n}{results\PYZus{}df} \PY{o}{=} \PY{n}{pd}\PY{o}{.}\PY{n}{DataFrame}\PY{p}{(}\PY{n}{results}\PY{p}{)}
\PY{n}{plt}\PY{o}{.}\PY{n}{figure}\PY{p}{(}\PY{n}{figsize}\PY{o}{=}\PY{p}{(}\PY{l+m+mi}{10}\PY{p}{,} \PY{l+m+mi}{6}\PY{p}{)}\PY{p}{)}
\PY{n}{plt}\PY{o}{.}\PY{n}{scatter}\PY{p}{(}\PY{n}{results\PYZus{}df}\PY{p}{[}\PY{l+s+s1}{\PYZsq{}}\PY{l+s+s1}{mse}\PY{l+s+s1}{\PYZsq{}}\PY{p}{]}\PY{p}{,} \PY{n}{results\PYZus{}df}\PY{p}{[}\PY{l+s+s1}{\PYZsq{}}\PY{l+s+s1}{num\PYZus{}vars}\PY{l+s+s1}{\PYZsq{}}\PY{p}{]}\PY{p}{)}
\PY{n}{plt}\PY{o}{.}\PY{n}{xlabel}\PY{p}{(}\PY{l+s+s1}{\PYZsq{}}\PY{l+s+s1}{MSE}\PY{l+s+s1}{\PYZsq{}}\PY{p}{)}
\PY{n}{plt}\PY{o}{.}\PY{n}{ylabel}\PY{p}{(}\PY{l+s+s1}{\PYZsq{}}\PY{l+s+s1}{Number of variables}\PY{l+s+s1}{\PYZsq{}}\PY{p}{)}
\PY{n}{plt}\PY{o}{.}\PY{n}{title}\PY{p}{(}\PY{l+s+s1}{\PYZsq{}}\PY{l+s+s1}{MSE relative to the number of variables used}\PY{l+s+s1}{\PYZsq{}}\PY{p}{)}
\PY{n}{plt}\PY{o}{.}\PY{n}{grid}\PY{p}{(}\PY{k+kc}{True}\PY{p}{)}
\PY{n}{plt}\PY{o}{.}\PY{n}{show}\PY{p}{(}\PY{p}{)}
\end{Verbatim}
\end{tcolorbox}

    \begin{center}
    \adjustimage{max size={0.9\linewidth}{0.9\paperheight}}{diabetes_files/diabetes_110_0.png}
    \end{center}
    { \hspace*{\fill} \\}
    
    After examining the labels for the three distinct diamond-shaped blobs
on the image above, I see that the first blob, with the least MSE
contains both \texttt{HbA1c\_level} and \texttt{blood\_glucose\_level},
the second blob contains one of the two, finally the third blob with the
most MSE contains neither.

It is clear, that these two variables are necessary, to improve
prediction. Moreover, they result in the lowest MSE when using two
variables only, an MSE of \(0.051\). This is the bottom dot of the first
diamond.

Adding \texttt{age} to them results in an MSE of \(0.049\) for three
variables, adding \texttt{bmi} results in an MSE of \(0.048\) for four
variables, adding \texttt{heart\_disease} results in an MSE of
\(0.0475\) for 5 variables, finally adding \texttt{hypertension} results
in \(0.0469\) for six variables.

Further variables don't decrease the MSE significantly.

    I will print these best linear regression models for all number of
variables:

    \begin{tcolorbox}[breakable, size=fbox, boxrule=1pt, pad at break*=1mm,colback=cellbackground, colframe=cellborder]
\prompt{In}{incolor}{277}{\boxspacing}
\begin{Verbatim}[commandchars=\\\{\}]
\PY{n}{min\PYZus{}mse\PYZus{}columns} \PY{o}{=} \PY{n}{results\PYZus{}df}\PY{o}{.}\PY{n}{groupby}\PY{p}{(}\PY{l+s+s1}{\PYZsq{}}\PY{l+s+s1}{num\PYZus{}vars}\PY{l+s+s1}{\PYZsq{}}\PY{p}{)}\PY{o}{.}\PY{n}{apply}\PY{p}{(}\PY{k}{lambda} \PY{n}{x}\PY{p}{:} \PY{n}{x}\PY{o}{.}\PY{n}{loc}\PY{p}{[}\PY{n}{x}\PY{p}{[}\PY{l+s+s1}{\PYZsq{}}\PY{l+s+s1}{mse}\PY{l+s+s1}{\PYZsq{}}\PY{p}{]}\PY{o}{.}\PY{n}{idxmin}\PY{p}{(}\PY{p}{)}\PY{p}{,} \PY{l+s+s1}{\PYZsq{}}\PY{l+s+s1}{columns}\PY{l+s+s1}{\PYZsq{}}\PY{p}{]}\PY{p}{)}
\end{Verbatim}
\end{tcolorbox}

    \begin{tcolorbox}[breakable, size=fbox, boxrule=1pt, pad at break*=1mm,colback=cellbackground, colframe=cellborder]
\prompt{In}{incolor}{278}{\boxspacing}
\begin{Verbatim}[commandchars=\\\{\}]
\PY{n}{n\PYZus{}vars} \PY{o}{=} \PY{l+m+mi}{1}
\PY{n}{cols} \PY{o}{=} \PY{n}{min\PYZus{}mse\PYZus{}columns}\PY{p}{[}\PY{n}{n\PYZus{}vars}\PY{p}{]}
\PY{n+nb}{print}\PY{p}{(}\PY{l+s+sa}{f}\PY{l+s+s2}{\PYZdq{}}\PY{l+s+s2}{Number of variables: }\PY{l+s+si}{\PYZob{}}\PY{n}{n\PYZus{}vars}\PY{l+s+si}{\PYZcb{}}\PY{l+s+s2}{\PYZdq{}}\PY{p}{)}
\PY{n+nb}{print}\PY{p}{(}\PY{p}{)}
\PY{n}{\PYZus{}} \PY{o}{=} \PY{n}{linear\PYZus{}regression}\PY{p}{(}\PY{n}{cols}\PY{p}{,} \PY{k+kc}{True}\PY{p}{)}
\end{Verbatim}
\end{tcolorbox}

    \begin{Verbatim}[commandchars=\\\{\}]
Number of variables: 1

Coefficients:
              Variable  Coefficient
0  blood\_glucose\_level     0.002731

Intercept:
    Variable  Coefficient
0  Intercept    -0.296492

Mean Squared Error:
0.05965061175137981
    \end{Verbatim}

    \begin{tcolorbox}[breakable, size=fbox, boxrule=1pt, pad at break*=1mm,colback=cellbackground, colframe=cellborder]
\prompt{In}{incolor}{279}{\boxspacing}
\begin{Verbatim}[commandchars=\\\{\}]
\PY{n}{n\PYZus{}vars} \PY{o}{=} \PY{l+m+mi}{2}
\PY{n}{cols} \PY{o}{=} \PY{n}{min\PYZus{}mse\PYZus{}columns}\PY{p}{[}\PY{n}{n\PYZus{}vars}\PY{p}{]}
\PY{n+nb}{print}\PY{p}{(}\PY{l+s+sa}{f}\PY{l+s+s2}{\PYZdq{}}\PY{l+s+s2}{Number of variables: }\PY{l+s+si}{\PYZob{}}\PY{n}{n\PYZus{}vars}\PY{l+s+si}{\PYZcb{}}\PY{l+s+s2}{\PYZdq{}}\PY{p}{)}
\PY{n+nb}{print}\PY{p}{(}\PY{p}{)}
\PY{n}{\PYZus{}} \PY{o}{=} \PY{n}{linear\PYZus{}regression}\PY{p}{(}\PY{n}{cols}\PY{p}{,} \PY{k+kc}{True}\PY{p}{)}
\end{Verbatim}
\end{tcolorbox}

    \begin{Verbatim}[commandchars=\\\{\}]
Number of variables: 2

Coefficients:
              Variable  Coefficient
0          HbA1c\_level     0.084451
1  blood\_glucose\_level     0.002379

Intercept:
    Variable  Coefficient
0  Intercept    -0.714191

Mean Squared Error:
0.051362749971308484
    \end{Verbatim}

    \begin{tcolorbox}[breakable, size=fbox, boxrule=1pt, pad at break*=1mm,colback=cellbackground, colframe=cellborder]
\prompt{In}{incolor}{280}{\boxspacing}
\begin{Verbatim}[commandchars=\\\{\}]
\PY{n}{n\PYZus{}vars} \PY{o}{=} \PY{l+m+mi}{3}
\PY{n}{cols} \PY{o}{=} \PY{n}{min\PYZus{}mse\PYZus{}columns}\PY{p}{[}\PY{n}{n\PYZus{}vars}\PY{p}{]}
\PY{n+nb}{print}\PY{p}{(}\PY{l+s+sa}{f}\PY{l+s+s2}{\PYZdq{}}\PY{l+s+s2}{Number of variables: }\PY{l+s+si}{\PYZob{}}\PY{n}{n\PYZus{}vars}\PY{l+s+si}{\PYZcb{}}\PY{l+s+s2}{\PYZdq{}}\PY{p}{)}
\PY{n+nb}{print}\PY{p}{(}\PY{p}{)}
\PY{n}{\PYZus{}} \PY{o}{=} \PY{n}{linear\PYZus{}regression}\PY{p}{(}\PY{n}{cols}\PY{p}{,} \PY{k+kc}{True}\PY{p}{)}
\end{Verbatim}
\end{tcolorbox}

    \begin{Verbatim}[commandchars=\\\{\}]
Number of variables: 3

Coefficients:
              Variable  Coefficient
1          HbA1c\_level     0.080259
0                  age     0.002462
2  blood\_glucose\_level     0.002255

Intercept:
    Variable  Coefficient
0  Intercept    -0.771513

Mean Squared Error:
0.048955894608910165
    \end{Verbatim}

    \begin{tcolorbox}[breakable, size=fbox, boxrule=1pt, pad at break*=1mm,colback=cellbackground, colframe=cellborder]
\prompt{In}{incolor}{281}{\boxspacing}
\begin{Verbatim}[commandchars=\\\{\}]
\PY{n}{n\PYZus{}vars} \PY{o}{=} \PY{l+m+mi}{4}
\PY{n}{cols} \PY{o}{=} \PY{n}{min\PYZus{}mse\PYZus{}columns}\PY{p}{[}\PY{n}{n\PYZus{}vars}\PY{p}{]}
\PY{n+nb}{print}\PY{p}{(}\PY{l+s+sa}{f}\PY{l+s+s2}{\PYZdq{}}\PY{l+s+s2}{Number of variables: }\PY{l+s+si}{\PYZob{}}\PY{n}{n\PYZus{}vars}\PY{l+s+si}{\PYZcb{}}\PY{l+s+s2}{\PYZdq{}}\PY{p}{)}
\PY{n+nb}{print}\PY{p}{(}\PY{p}{)}
\PY{n}{\PYZus{}} \PY{o}{=} \PY{n}{linear\PYZus{}regression}\PY{p}{(}\PY{n}{cols}\PY{p}{,} \PY{k+kc}{True}\PY{p}{)}
\end{Verbatim}
\end{tcolorbox}

    \begin{Verbatim}[commandchars=\\\{\}]
Number of variables: 4

Coefficients:
              Variable  Coefficient
2          HbA1c\_level     0.079186
1                  bmi     0.004354
3  blood\_glucose\_level     0.002223
0                  age     0.001940

Intercept:
    Variable  Coefficient
0  Intercept    -0.859603

Mean Squared Error:
0.04820938328094974
    \end{Verbatim}

    \begin{tcolorbox}[breakable, size=fbox, boxrule=1pt, pad at break*=1mm,colback=cellbackground, colframe=cellborder]
\prompt{In}{incolor}{282}{\boxspacing}
\begin{Verbatim}[commandchars=\\\{\}]
\PY{n}{n\PYZus{}vars} \PY{o}{=} \PY{l+m+mi}{5}
\PY{n}{cols} \PY{o}{=} \PY{n}{min\PYZus{}mse\PYZus{}columns}\PY{p}{[}\PY{n}{n\PYZus{}vars}\PY{p}{]}
\PY{n+nb}{print}\PY{p}{(}\PY{l+s+sa}{f}\PY{l+s+s2}{\PYZdq{}}\PY{l+s+s2}{Number of variables: }\PY{l+s+si}{\PYZob{}}\PY{n}{n\PYZus{}vars}\PY{l+s+si}{\PYZcb{}}\PY{l+s+s2}{\PYZdq{}}\PY{p}{)}
\PY{n+nb}{print}\PY{p}{(}\PY{p}{)}
\PY{n}{\PYZus{}} \PY{o}{=} \PY{n}{linear\PYZus{}regression}\PY{p}{(}\PY{n}{cols}\PY{p}{,} \PY{k+kc}{True}\PY{p}{)}
\end{Verbatim}
\end{tcolorbox}

    \begin{Verbatim}[commandchars=\\\{\}]
Number of variables: 5

Coefficients:
              Variable  Coefficient
1        heart\_disease     0.147063
3          HbA1c\_level     0.078263
2                  bmi     0.004424
4  blood\_glucose\_level     0.002196
0                  age     0.001681

Intercept:
    Variable  Coefficient
0  Intercept    -0.847087

Mean Squared Error:
0.04750629195281252
    \end{Verbatim}

    \begin{tcolorbox}[breakable, size=fbox, boxrule=1pt, pad at break*=1mm,colback=cellbackground, colframe=cellborder]
\prompt{In}{incolor}{283}{\boxspacing}
\begin{Verbatim}[commandchars=\\\{\}]
\PY{n}{n\PYZus{}vars} \PY{o}{=} \PY{l+m+mi}{6}
\PY{n}{cols} \PY{o}{=} \PY{n}{min\PYZus{}mse\PYZus{}columns}\PY{p}{[}\PY{n}{n\PYZus{}vars}\PY{p}{]}
\PY{n+nb}{print}\PY{p}{(}\PY{l+s+sa}{f}\PY{l+s+s2}{\PYZdq{}}\PY{l+s+s2}{Number of variables: }\PY{l+s+si}{\PYZob{}}\PY{n}{n\PYZus{}vars}\PY{l+s+si}{\PYZcb{}}\PY{l+s+s2}{\PYZdq{}}\PY{p}{)}
\PY{n+nb}{print}\PY{p}{(}\PY{p}{)}
\PY{n}{\PYZus{}} \PY{o}{=} \PY{n}{linear\PYZus{}regression}\PY{p}{(}\PY{n}{cols}\PY{p}{,} \PY{k+kc}{True}\PY{p}{)}
\end{Verbatim}
\end{tcolorbox}

    \begin{Verbatim}[commandchars=\\\{\}]
Number of variables: 6

Coefficients:
              Variable  Coefficient
2        heart\_disease     0.137909
1         hypertension     0.105400
4          HbA1c\_level     0.077108
3                  bmi     0.004134
5  blood\_glucose\_level     0.002167
0                  age     0.001437

Intercept:
    Variable  Coefficient
0  Intercept    -0.826001

Mean Squared Error:
0.04699444731206997
    \end{Verbatim}

    In conclusion, the blood level variables seem the most important and
also most difficult to measure. Adding trivially measurable variables,
such as the \texttt{age} and \texttt{bmi} improves the MSE by \(0.003\),
so by default I would use the 4 variable model.

If history of heart-related issues is known, then I would also measure
\texttt{hypertension}, which is directly related to heart disease and
use the 6 variable model for diabetes prediction.


    % Add a bibliography block to the postdoc
    
    
    
\end{document}
